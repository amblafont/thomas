\section{Inserters: le retour}
Ce que je vais dire ce generalise sans pb a n'importe quelle 2-cat avec pullback.
En fait la l'idee c'est de dire que $\Cat^\rightarrow \xrightarrow{cod} \Cat$
est une 2-bifibration, avec le pullback donne par le pullback. Consequence: 
un inserter dans une 2-cat fibre (qui est donc un inserter vertical) est aussi un 
inserter dans $\Cat^→$ (par une generalisation 2-categorique de \cite[Corollary 4.3]{grayfib}).
Du coup, il suffit de montrer que les limites calculees pointwise se relevent en un 2-foncteur:
\begin{lemma}
    If $A \xrightarrow{P} \VInsDiag$ (cf la section inserter vertical pour la def
    de $\VInsDiag$, a la difference qu'on enleve la partie verticale, et on peut remplacer $\Cat$ par n'importe quel 2-cat) is a 2-functor such that
    $P(a)$ has an inserter $L(a)$, then $L$ uniquely lifts to a 2-functor $A \rightarrow E$
\end{lemma}
Pour faire la preuve de ca, on fait comme decrit dans la section sur les inserters verticaux (en enlevant la partie verticale).
Mais la difference, c'est qu'on va faire a la fin  le pullback de 
la 2-opfibration $\VInsCone → \VInsDiag$ le long de $A \rightarrow \VInsDiag $
avant d'appliquer le lemme \ref{lem:twofib-section}.

%%% Local Variables:
%%% TeX-master: "../main.tex"
%%% End:
