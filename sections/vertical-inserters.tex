\section{Vertical Inserters}

\todot{Cette section devrait remplacer celle sur les inserters. En fait non! Cf la simplification dans la section suivante}
L'idee de cette section c'etait de montrer qu'on a une split 2-opfibration entre les cones et les diagrammes.
$  \VInsCone \rightarrow \VInsDiag$. Et de tirer parti du lemme suivant:
\begin{lemma}
  \label{lem:twofib-section}
  Given a split 2-fibration, if each fiber 2-category has a 1-initial object, 
  then the fibration has a (unique?) section (as a 2-functor) selecting each initial object.
\end{lemma}
\begin{remark}
  C'est assez etonnant que la propriete universelle 1-categorique suffise a induire un 2-foncteur.
\end{remark}
Et ensuite du coup on compose la section $\VInsDiag → \VInsCone$ avec $\VInsCone → Cat$.

\begin{definition}
  We define a 2-category $\VInsDiag$ of vertical inserter diagrams (abbreviated as
  vins-diagrams) which is a wide, locally full sub 2-category 
  of the 2-category of functors from 
  $\begin{tikzcd}
   ⋅ \doublerightarrow{r}{}{}& ⋅ \ar[r] & ⋅  
  \end{tikzcd}
  $
  to $\Cat$, lax natural transformations and modifications~\cite{twodimcat}.
   An object is given by a diagram of three functors 
   $
   \begin{tikzcd}
   A \doublerightarrow{r}{F}{G} & B \ar[r,"p"] & C.
   \end{tikzcd}
   $
   A 1-cell between $(F,G,p)$ and $(F',G',p')$ is given 
   by three functors $(A \xrightarrow{H_A} A', B\xrightarrow{H_B} B', C\xrightarrow{H_C} C')$ such that 
   $H_C 	
   ∘
   p = p' 	
   ∘
    H_B$ and $H_B ∘ G = G' ∘ H_A$, together with
   a natural transformation 
   $  F' H_A \xrightarrow{h_F}  H_BF$ 
   as summarised in the following diagram
   \[
   % YADE DIAGRAM diagrams/vins-diag-1-cell.json
% GENERATED LATEX
\begin{tikzpicture}[every node/.style={inner sep=2pt,outer sep=0pt,anchor=base,text height=1.2ex, text depth=0.25ex}] 
\node (0) at (3.8095238095238093em, -3.8095238095238093em) {$A$} ; 
\node (1) at (11.428571428571429em, -3.8095238095238093em) {$B$} ; 
\node (2) at (11.428571428571429em, -11.428571428571429em) {$B'$} ; 
\node (3) at (3.8095238095238093em, -11.428571428571429em) {$A'$} ; 
\node (4) at (19.047619047619047em, -3.8095238095238093em) {$C$} ; 
\node (5) at (19.047619047619047em, -11.428571428571429em) {$C'$} ; 
\path 
(0) to[fore, black,->, curve={ratio=-0.1}, ] node[coordinate](6){} (1) 
(0) to[fore, black,->, curve={ratio=0.1}, ] node[coordinate](7){} (1) 
(1) to[fore, black,->, ] node[coordinate](8){} (2) 
(0) to[fore, black,->, ] node[coordinate](9){} (3) 
(3) to[fore, black,->, curve={ratio=-0.1}, ] node[coordinate](10){} (2) 
(3) to[fore, black,->, curve={ratio=0.1}, ] node[coordinate](11){} (2) 
(1) to[fore, black,->, ] node[coordinate](12){} (4) 
(2) to[fore, black,->, ] node[coordinate](13){} (5) 
(4) to[fore, black,->, ] node[coordinate](14){} (5) 
(9) to[fore, black,->, cell=0.2, curve={ratio=-0.1}, ] node[coordinate](15){} (8) 
; 
\path[->] 
(0) edge["${\scriptstyle F}$", pos=0.5, fore, black,->, curve={ratio=-0.1}, ] (1) 
(0) edge["${\scriptstyle G}$"', pos=0.5, fore, black,->, curve={ratio=0.1}, ] (1) 
(1) edge["${\scriptstyle H_B}$", pos=0.5, fore, black,->, ] (2) 
(0) edge["${\scriptstyle H_A}$"', pos=0.5, fore, black,->, ] (3) 
(3) edge["${\scriptstyle F'}$", pos=0.5, fore, black,->, curve={ratio=-0.1}, ] (2) 
(3) edge["${\scriptstyle G'}$"', pos=0.5, fore, black,->, curve={ratio=0.1}, ] (2) 
(1) edge["${\scriptstyle p}$", pos=0.5, fore, black,->, ] (4) 
(2) edge["${\scriptstyle p'}$"', pos=0.5, fore, black,->, ] (5) 
(4) edge["${\scriptstyle H_C}$", pos=0.5, fore, black,->, ] (5) 
(9) edge["${\scriptstyle h_F}$", pos=0.5, fore, black,->, cell=0.2, curve={ratio=-0.1}, ] (8) 
; 
\end{tikzpicture}
% END OF GENERATED LATEX
\]
% En termes de string diagram (bottom up, left right)
% h_F 
%   F  H_B
%   H_A F'

% h_G 
%   H_A G'
%   G H_B

%h_p est une egalite
A 2-cell between $(H_A, H_B, H_C, h_F)$ and
$(H_A', H_B', H_C', h_F')$ is given by three natural transformations
$(
  H_A \xrightarrow{h_A}H_A,
  H_B \xrightarrow{h_B}H_B,
  H_C \xrightarrow{h_C}H_C
)$ such that the following equalities hold:
\[
% YADE DIAGRAM diagrams/vins-2cell.json  
% GENERATED LATEX
\begin{tikzpicture}[every node/.style={inner sep=2pt,outer sep=0pt,anchor=base,text height=1.2ex, text depth=0.25ex}] 
\node (0) at (19.047619047619047em, -3.8095238095238093em) {$A$} ; 
\node (1) at (19.047619047619047em, -11.428571428571429em) {$A'$} ; 
\node (2) at (26.666666666666668em, -3.8095238095238093em) {$B$} ; 
\node (3) at (26.666666666666668em, -11.428571428571429em) {$B'$} ; 
\node (4) at (30.380975882212322em, -7.553375244140625em) {$=$} ; 
\node (5) at (34.285714285714285em, -3.8095238095238093em) {$A$} ; 
\node (6) at (34.285714285714285em, -11.428571428571429em) {$A'$} ; 
\node (7) at (41.904761904761905em, -3.8095238095238093em) {$B$} ; 
\node (8) at (41.904761904761905em, -11.428571428571429em) {$B'$} ; 
\node (9) at (19.047619047619047em, -15.258347942715599em) {$A$} ; 
\node (10) at (19.047619047619047em, -22.87739556176322em) {$A'$} ; 
\node (11) at (26.666666666666668em, -15.258347942715599em) {$B$} ; 
\node (12) at (26.666666666666668em, -22.87739556176322em) {$B'$} ; 
\node (13) at (30.380975882212322em, -19.002199377332413em) {$=$} ; 
\node (14) at (34.285714285714285em, -15.258347942715599em) {$A$} ; 
\node (15) at (34.285714285714285em, -22.87739556176322em) {$A'$} ; 
\node (16) at (41.904761904761905em, -15.258347942715599em) {$B$} ; 
\node (17) at (41.904761904761905em, -22.87739556176322em) {$B'$} ; 
\node (18) at (19.047619047619047em, -26.524648757207963em) {$B$} ; 
\node (19) at (19.047619047619047em, -34.14369637625558em) {$B'$} ; 
\node (20) at (26.666666666666668em, -26.524648757207963em) {$C$} ; 
\node (21) at (26.666666666666668em, -34.14369637625558em) {$C'$} ; 
\node (22) at (30.380975882212322em, -30.268500191824778em) {$=$} ; 
\node (23) at (34.285714285714285em, -26.524648757207963em) {$B$} ; 
\node (24) at (34.285714285714285em, -34.14369637625558em) {$B'$} ; 
\node (25) at (41.904761904761905em, -26.524648757207963em) {$C$} ; 
\node (26) at (41.904761904761905em, -34.14369637625558em) {$C'$} ; 
\path 
(0) to[fore, black,->, curve={ratio=0.2}, ] node[coordinate](27){} (1) 
(0) to[fore, black,->, curve={ratio=-0.2}, ] node[coordinate](28){} (1) 
(0) to[fore, black,->, ] node[coordinate](29){} (2) 
(1) to[fore, black,->, ] node[coordinate](30){} (3) 
(2) to[fore, black,->, curve={ratio=-0.1}, ] node[coordinate](31){} (3) 
(5) to[fore, black,->, curve={ratio=0.2}, ] node[coordinate](32){} (6) 
(5) to[fore, black,->, ] node[coordinate](33){} (7) 
(6) to[fore, black,->, ] node[coordinate](34){} (8) 
(7) to[fore, black,->, curve={ratio=0.2}, ] node[coordinate](35){} (8) 
(7) to[fore, black,->, curve={ratio=-0.2}, ] node[coordinate](36){} (8) 
(9) to[fore, black,->, curve={ratio=0.2}, ] node[coordinate](37){} (10) 
(9) to[fore, black,->, curve={ratio=-0.2}, ] node[coordinate](38){} (10) 
(9) to[fore, black,->, ] node[coordinate](39){} (11) 
(10) to[fore, black,->, ] node[coordinate](40){} (12) 
(11) to[fore, black,->, curve={ratio=-0.1}, ] node[coordinate](41){} (12) 
(14) to[fore, black,->, curve={ratio=0.2}, ] node[coordinate](42){} (15) 
(14) to[fore, black,->, ] node[coordinate](43){} (16) 
(15) to[fore, black,->, ] node[coordinate](44){} (17) 
(16) to[fore, black,->, curve={ratio=0.2}, ] node[coordinate](45){} (17) 
(16) to[fore, black,->, curve={ratio=-0.2}, ] node[coordinate](46){} (17) 
(18) to[fore, black,->, curve={ratio=0.2}, ] node[coordinate](47){} (19) 
(18) to[fore, black,->, curve={ratio=-0.2}, ] node[coordinate](48){} (19) 
(18) to[fore, black,->, ] node[coordinate](49){} (20) 
(19) to[fore, black,->, ] node[coordinate](50){} (21) 
(20) to[fore, black,->, curve={ratio=-0.1}, ] node[coordinate](51){} (21) 
(23) to[fore, black,->, curve={ratio=0.2}, ] node[coordinate](52){} (24) 
(23) to[fore, black,->, ] node[coordinate](53){} (25) 
(24) to[fore, black,->, ] node[coordinate](54){} (26) 
(25) to[fore, black,->, curve={ratio=0.2}, ] node[coordinate](55){} (26) 
(25) to[fore, black,->, curve={ratio=-0.2}, ] node[coordinate](56){} (26) 
(27) to[fore, black,->, cell=0.2, ] node[coordinate](57){} (28) 
(28) to[fore, black,->, cell=0.2, ] node[coordinate](58){} (31) 
(32) to[fore, black,->, cell=0.2, ] node[coordinate](59){} (35) 
(35) to[fore, black,->, cell=0.2, ] node[coordinate](60){} (36) 
(37) to[fore, black,->, cell=0.2, ] node[coordinate](61){} (38) 
(45) to[fore, black,->, cell=0.2, ] node[coordinate](62){} (46) 
(47) to[fore, black,->, cell=0.2, ] node[coordinate](63){} (48) 
(55) to[fore, black,->, cell=0.2, ] node[coordinate](64){} (56) 
; 
\path[->] 
(0) edge["${\scriptstyle H_A}$"', pos=0.5, fore, black,->, curve={ratio=0.2}, ] (1) 
(0) edge["${\scriptstyle H_A'}$", pos=0.20000000000000004, fore, black,->, curve={ratio=-0.2}, ] (1) 
(0) edge["${\scriptstyle F}$", pos=0.5, fore, black,->, ] (2) 
(1) edge["${\scriptstyle F'}$"', pos=0.5, fore, black,->, ] (3) 
(2) edge["${\scriptstyle H_B'}$", pos=0.5, fore, black,->, curve={ratio=-0.1}, ] (3) 
(5) edge["${\scriptstyle H_A}$"', pos=0.5, fore, black,->, curve={ratio=0.2}, ] (6) 
(5) edge["${\scriptstyle F}$", pos=0.5, fore, black,->, ] (7) 
(6) edge["${\scriptstyle F'}$"', pos=0.5, fore, black,->, ] (8) 
(7) edge["${\scriptstyle H_B}$"', pos=0.30000000000000004, fore, black,->, curve={ratio=0.2}, ] (8) 
(7) edge["${\scriptstyle H_B'}$", pos=0.5, fore, black,->, curve={ratio=-0.2}, ] (8) 
(9) edge["${\scriptstyle H_A}$"', pos=0.5, fore, black,->, curve={ratio=0.2}, ] (10) 
(9) edge["${\scriptstyle H_A'}$", pos=0.5, fore, black,->, curve={ratio=-0.2}, ] (10) 
(9) edge["${\scriptstyle G}$", pos=0.5, fore, black,->, ] (11) 
(10) edge["${\scriptstyle G'}$"', pos=0.5, fore, black,->, ] (12) 
(11) edge["${\scriptstyle H_B'}$", pos=0.5, fore, black,->, curve={ratio=-0.1}, ] (12) 
(14) edge["${\scriptstyle H_A}$"', pos=0.5, fore, black,->, curve={ratio=0.2}, ] (15) 
(14) edge["${\scriptstyle G}$", pos=0.5, fore, black,->, ] (16) 
(15) edge["${\scriptstyle G'}$"', pos=0.5, fore, black,->, ] (17) 
(16) edge["${\scriptstyle H_B}$"', pos=0.5, fore, black,->, curve={ratio=0.2}, ] (17) 
(16) edge["${\scriptstyle H_B'}$", pos=0.5, fore, black,->, curve={ratio=-0.2}, ] (17) 
(18) edge["${\scriptstyle H_B}$"', pos=0.5, fore, black,->, curve={ratio=0.2}, ] (19) 
(18) edge["${\scriptstyle H_B'}$", pos=0.5, fore, black,->, curve={ratio=-0.2}, ] (19) 
(18) edge["${\scriptstyle p}$", pos=0.5, fore, black,->, ] (20) 
(19) edge["${\scriptstyle p'}$"', pos=0.5, fore, black,->, ] (21) 
(20) edge["${\scriptstyle H_C'}$", pos=0.5, fore, black,->, curve={ratio=-0.1}, ] (21) 
(23) edge["${\scriptstyle H_B}$"', pos=0.5, fore, black,->, curve={ratio=0.2}, ] (24) 
(23) edge["${\scriptstyle p}$", pos=0.5, fore, black,->, ] (25) 
(24) edge["${\scriptstyle p'}$"', pos=0.5, fore, black,->, ] (26) 
(25) edge["${\scriptstyle H_C}$"', pos=0.5, fore, black,->, curve={ratio=0.2}, ] (26) 
(25) edge["${\scriptstyle H_C'}$", pos=0.5, fore, black,->, curve={ratio=-0.2}, ] (26) 
(27) edge["${\scriptstyle h_A}$", pos=0.5, fore, black,->, cell=0.2, ] (28) 
(28) edge["${\scriptstyle h_F'}$", pos=0.5, fore, black,->, cell=0.2, ] (31) 
(32) edge["${\scriptstyle h_F}$", pos=0.5, fore, black,->, cell=0.2, ] (35) 
(35) edge["${\scriptstyle h_B}$", pos=0.5, fore, black,->, cell=0.2, ] (36) 
(37) edge["${\scriptstyle h_A}$", pos=0.5, fore, black,->, cell=0.2, ] (38) 
(45) edge["${\scriptstyle h_B}$", pos=0.5, fore, black,->, cell=0.2, ] (46) 
(47) edge["${\scriptstyle h_B}$", pos=0.5, fore, black,->, cell=0.2, ] (48) 
(55) edge["${\scriptstyle h_C}$", pos=0.5, fore, black,->, cell=0.2, ] (56) 
; 
\end{tikzpicture}
% END OF GENERATED LATEX
\]

\end{definition}
\begin{definition}
  We define a 2-category $\VInsCone$ of vertical inserter cones (abbreviated as
  vins-cones) which is a wide, locally full sub 2-category 
  of the 2-category of functors from 
  
  $
  % YADE DIAGRAM diagrams/vins-cone-shape.json
% GENERATED LATEX
\begin{tikzpicture}[every node/.style={inner sep=2pt,outer sep=0pt,anchor=base,text height=1.2ex, text depth=0.25ex}] 
\node (0) at (10.714285714285714em, -5.9523809523809526em) {$X$} ; 
\node (1) at (13.095238095238095em, -3.5714285714285716em) {$A$} ; 
\node (2) at (13.095238095238095em, -8.333333333333334em) {$A$} ; 
\node (3) at (15.476190476190476em, -5.9523809523809526em) {$B$} ; 
\node (4) at (17.857142857142858em, -5.9523809523809526em) {$C$} ; 
\path 
(0) to[fore, black,->, curve={ratio=-0.1}, ] node[coordinate](5){} (1) 
(0) to[fore, black,->, curve={ratio=0.1}, ] node[coordinate](6){} (2) 
(2) to[fore, black,->, curve={ratio=0.1}, ] node[coordinate](7){} (3) 
(1) to[fore, black,->, curve={ratio=-0.1}, ] node[coordinate](8){} (3) 
(3) to[fore, black,->, ] node[coordinate](9){} (4) 
(1) to[fore, black,->, cell=0.2, ] node[coordinate](10){} (2) 
; 
\path[->] 
(0) edge["${\scriptstyle x}$", pos=0.5, fore, black,->, curve={ratio=-0.1}, ] (1) 
(0) edge["${\scriptstyle x}$"', pos=0.5, fore, black,->, curve={ratio=0.1}, ] (2) 
(2) edge["${\scriptstyle G}$"', pos=0.5, fore, black,->, curve={ratio=0.1}, ] (3) 
(1) edge["${\scriptstyle F}$", pos=0.5, fore, black,->, curve={ratio=-0.1}, ] (3) 
(3) edge["${\scriptstyle p}$", pos=0.5, fore, black,->, ] (4) 
(1) edge["${\scriptstyle \alpha}$", pos=0.5, fore, black,->, cell=0.2, ] (2) 
; 
\end{tikzpicture},
% END OF GENERATED LATEX
  $
  where $p ⋅ α$ is an identity 2-cell, 
  to $\Cat$, lax natural transformations and modifications.
   An object is given by a vertical inserter diagram 
   $
   \begin{tikzcd}
   A \doublerightarrow{r}{F}{G} & B \ar[r,"p"] & C
   \end{tikzcd}
   $
   and a \emph{cone} over it, that is:
   \begin{itemize}
    \item a category $X$;
    \item a functor $X \xrightarrow{x}A$;
    \item a natural transformation  $ F · x \xrightarrow{α} G · x$ such that 
     $p ⋅ α$ is an identity natural transformation
     (which entails that $p∘ F∘ x = p ∘ G ∘ x$).
   \end{itemize}
   A 1-cell between $(F,G,p,α)$ and $(F',G',p',α')$ is given 
   by a 1-cell $(H_A,H_B,H_C,h_F)$ between 
   the underlying vins-diagrams and a functor 
   $X\xrightarrow{H_X} X'$ such that 
   $H_X ∘ x' = x ∘ H_A$.
   as summarised in the following diagram, and additionaly such that 
   Equation~\eqref{eq:vins-mor-cones} holds.
   \todot{Il doit manquer des conditions 
   relatives au fait que $p⋅ α $ est une identite}.
   \[
   % YADE DIAGRAM diagrams/vins-1-cell.json
% GENERATED LATEX
\begin{tikzpicture}[every node/.style={inner sep=2pt,outer sep=0pt,anchor=base,text height=1.2ex, text depth=0.25ex}] 
\node (0) at (11.428571428571429em, -3.8095238095238093em) {$A$} ; 
\node (1) at (19.047619047619047em, -3.8095238095238093em) {$B$} ; 
\node (2) at (19.047619047619047em, -11.428571428571429em) {$B'$} ; 
\node (3) at (11.428571428571429em, -11.428571428571429em) {$A'$} ; 
\node (4) at (26.666666666666668em, -3.8095238095238093em) {$C$} ; 
\node (5) at (26.666666666666668em, -11.428571428571429em) {$C'$} ; 
\node (6) at (3.8095238095238093em, -3.8095238095238093em) {$X$} ; 
\node (7) at (3.8095238095238093em, -11.428571428571429em) {$X'$} ; 
\path 
(0) to[fore, black,->, curve={ratio=-0.1}, ] node[coordinate](8){} (1) 
(0) to[fore, black,->, curve={ratio=0.1}, ] node[coordinate](9){} (1) 
(1) to[fore, black,->, ] node[coordinate](10){} (2) 
(0) to[fore, black,->, ] node[coordinate](11){} (3) 
(3) to[fore, black,->, curve={ratio=-0.1}, ] node[coordinate](12){} (2) 
(3) to[fore, black,->, curve={ratio=0.1}, ] node[coordinate](13){} (2) 
(1) to[fore, black,->, ] node[coordinate](14){} (4) 
(2) to[fore, black,->, ] node[coordinate](15){} (5) 
(4) to[fore, black,->, ] node[coordinate](16){} (5) 
(6) to[fore, black,->, ] node[coordinate](17){} (0) 
(6) to[fore, black,->, ] node[coordinate](18){} (7) 
(7) to[fore, black,->, ] node[coordinate](19){} (3) 
(11) to[fore, black,->, cell=0.2, curve={ratio=-0.2}, ] node[coordinate](20){} (10) 
; 
\path[->] 
(0) edge["${\scriptstyle F}$", pos=0.5, fore, black,->, curve={ratio=-0.1}, ] (1) 
(0) edge["${\scriptstyle G}$"', pos=0.5, fore, black,->, curve={ratio=0.1}, ] (1) 
(1) edge["${\scriptstyle H_B}$", pos=0.5, fore, black,->, ] (2) 
(0) edge["${\scriptstyle H_A}$"', pos=0.5, fore, black,->, ] (3) 
(3) edge["${\scriptstyle F'}$", pos=0.5, fore, black,->, curve={ratio=-0.1}, ] (2) 
(3) edge["${\scriptstyle G'}$"', pos=0.5, fore, black,->, curve={ratio=0.1}, ] (2) 
(1) edge["${\scriptstyle p}$", pos=0.5, fore, black,->, ] (4) 
(2) edge["${\scriptstyle p'}$"', pos=0.5, fore, black,->, ] (5) 
(4) edge["${\scriptstyle H_C}$", pos=0.5, fore, black,->, ] (5) 
(6) edge["${\scriptstyle x}$", pos=0.5, fore, black,->, ] (0) 
(6) edge["${\scriptstyle H_X}$"', pos=0.5, fore, black,->, ] (7) 
(7) edge["${\scriptstyle x'}$"', pos=0.5, fore, black,->, ] (3) 
(11) edge["${\scriptstyle h_F}$", pos=0.5, fore, black,->, cell=0.2, curve={ratio=-0.2}, ] (10) 
; 
\end{tikzpicture}
% END OF GENERATED LATEX
\]
\begin{equation}
  \label{eq:vins-mor-cones}
  % YADE DIAGRAM diagrams/vins-cone-1-cell-compat.json
% GENERATED LATEX
\begin{tikzpicture}[every node/.style={inner sep=2pt,outer sep=0pt,anchor=base,text height=1.2ex, text depth=0.25ex}] 
\node (0) at (20.476213977450417em, -0.7994464692615327em) {(naturalite de la transfo lax)} ; 
\node (1) at (10em, -10em) {$X$} ; 
\node (2) at (16.666666666666668em, -3.3333333333333335em) {$A$} ; 
\node (3) at (23.333333333333332em, -10em) {$B$} ; 
\node (4) at (16.666666666666668em, -10em) {$A$} ; 
\node (5) at (10em, -16.666666666666668em) {$X'$} ; 
\node (6) at (16.666666666666668em, -16.666666666666668em) {$A'$} ; 
\node (7) at (23.333333333333332em, -16.666666666666668em) {$B'$} ; 
\node (8) at (30em, -3.3333333333333335em) {$X$} ; 
\node (9) at (36.666666666666664em, -3.3333333333333335em) {$A$} ; 
\node (10) at (43.333333333333336em, -3.3333333333333335em) {$B$} ; 
\node (11) at (43.333333333333336em, -10em) {$B'$} ; 
\node (12) at (36.666666666666664em, -10em) {$A'$} ; 
\node (13) at (30em, -10em) {$X'$} ; 
\node (14) at (36.666666666666664em, -16.666666666666668em) {$A'$} ; 
\node (15) at (26.666690167926607em, -10em) {$=$} ; 
\path 
(1) to[fore, black,->, curve={ratio=-0.30000000000000004}, ] node[coordinate](16){} (2) 
(2) to[fore, black,->, curve={ratio=-0.30000000000000004}, ] node[coordinate](17){} (3) 
(1) to[fore, black,->, ] node[coordinate](18){} (4) 
(4) to[fore, black,->, ] node[coordinate](19){} (3) 
(4) to[fore, black,->, cell=0.2, ] node[coordinate](20){} (2) 
(1) to[fore, black,->, ] node[coordinate](21){} (5) 
(4) to[fore, black,->, ] node[coordinate](22){} (6) 
(5) to[fore, black,->, ] node[coordinate](23){} (6) 
(3) to[fore, black,->, ] node[coordinate](24){} (7) 
(6) to[fore, black,->, ] node[coordinate](25){} (7) 
(8) to[fore, black,->, ] node[coordinate](26){} (9) 
(9) to[fore, black,->, ] node[coordinate](27){} (10) 
(10) to[fore, black,->, ] node[coordinate](28){} (11) 
(9) to[fore, black,->, ] node[coordinate](29){} (12) 
(12) to[fore, black,->, ] node[coordinate](30){} (11) 
(8) to[fore, black,->, ] node[coordinate](31){} (13) 
(13) to[fore, black,->, ] node[coordinate](32){} (12) 
(13) to[fore, black,->, curve={ratio=0.2}, ] node[coordinate](33){} (14) 
(14) to[fore, black,->, curve={ratio=0.2}, ] node[coordinate](34){} (11) 
(14) to[fore, black,->, cell=0.2, ] node[coordinate](35){} (12) 
(25) to[fore, black,->, cell=0.2, ] node[coordinate](36){} (19) 
; 
\path[->] 
(1) edge["${\scriptstyle x}$", pos=0.5, fore, black,->, curve={ratio=-0.30000000000000004}, ] (2) 
(2) edge["${\scriptstyle G}$", pos=0.5, fore, black,->, curve={ratio=-0.30000000000000004}, ] (3) 
(1) edge["${\scriptstyle x}$"', pos=0.5, fore, black,->, ] (4) 
(4) edge["${\scriptstyle F}$", pos=0.5, fore, black,->, ] (3) 
(4) edge["${\scriptstyle \alpha}$"', pos=0.5, fore, black,->, cell=0.2, ] (2) 
(1) edge["${\scriptstyle H_X}$"', pos=0.5, fore, black,->, ] (5) 
(4) edge["${\scriptstyle H_A}$", pos=0.5, fore, black,->, ] (6) 
(5) edge["${\scriptstyle x'}$", pos=0.5, fore, black,->, ] (6) 
(3) edge["${\scriptstyle H_B}$", pos=0.5, fore, black,->, ] (7) 
(6) edge["${\scriptstyle F'}$"', pos=0.5, fore, black,->, ] (7) 
(8) edge["${\scriptstyle x}$", pos=0.5, fore, black,->, ] (9) 
(9) edge["${\scriptstyle G}$", pos=0.5, fore, black,->, ] (10) 
(10) edge["${\scriptstyle H_B}$", pos=0.5, fore, black,->, ] (11) 
(9) edge["${\scriptstyle H_A}$", pos=0.5, fore, black,->, ] (12) 
(12) edge["${\scriptstyle G'}$", pos=0.5, fore, black,->, ] (11) 
(8) edge["${\scriptstyle H_X}$", pos=0.5, fore, black,->, ] (13) 
(13) edge["${\scriptstyle x'}$", pos=0.5, fore, black,->, ] (12) 
(13) edge["${\scriptstyle x'}$", pos=0.5, fore, black,->, curve={ratio=0.2}, ] (14) 
(14) edge["${\scriptstyle F'}$"', pos=0.5, fore, black,->, curve={ratio=0.2}, ] (11) 
(14) edge["${\scriptstyle \alpha'}$", pos=0.5, fore, black,->, cell=0.2, ] (12) 
(25) edge["${\scriptstyle h_F}$", pos=0.5, fore, black,->, cell=0.2, ] (19) 
; 
\end{tikzpicture}
% END OF GENERATED LATEX
\end{equation}
% En termes de string diagram (bottom up, left right)
% h_F 
%   F  H_B
%   H_A F'

% h_G 
%   H_A G'
%   G H_B

%h_p est une egalite
A 2-cell between $(H_X,H_A, H_B, H_C, h_F)$ and
$(H_X',H_A', H_B', H_C', h_F')$ is given
by a 2-cell $(h_A,h_B,h_C)$ in $\VInsDiag$ and a natural transformation
$H_X \xrightarrow{h_X} H_X'$ such that
 such that the following equalities hold:
\[
% YADE DIAGRAM diagrams/vins-cone-2cell.json  
% GENERATED LATEX
\begin{tikzpicture}[every node/.style={inner sep=2pt,outer sep=0pt,anchor=base,text height=1.2ex, text depth=0.25ex}] 
\node (0) at (19.047619047619047em, -3.8095238095238093em) {$X$} ; 
\node (1) at (19.047619047619047em, -11.428571428571429em) {$X'$} ; 
\node (2) at (26.666666666666668em, -3.8095238095238093em) {$A$} ; 
\node (3) at (26.666666666666668em, -11.428571428571429em) {$A'$} ; 
\node (4) at (30.380975882212322em, -7.553375244140625em) {$=$} ; 
\node (5) at (34.285714285714285em, -3.8095238095238093em) {$X$} ; 
\node (6) at (34.285714285714285em, -11.428571428571429em) {$X'$} ; 
\node (7) at (41.904761904761905em, -3.8095238095238093em) {$A$} ; 
\node (8) at (41.904761904761905em, -11.428571428571429em) {$A'$} ; 
\node (9) at (11.428571428571429em, -19.047619047619047em) {$X$} ; 
\node (10) at (11.428571428571429em, -32.01899283272879em) {$X'$} ; 
\node (11) at (16.396140870593843em, -26.71811312720889em) {$A$} ; 
\node (12) at (19.047619047619047em, -19.047619047619047em) {$A$} ; 
\node (13) at (21.681855156308128em, -26.670494079589844em) {$B$} ; 
\node (14) at (19.047619047619047em, -32.01899283272879em) {$A'$} ; 
\node (15) at (34.285714285714285em, -19.047619047619047em) {$X$} ; 
\node (16) at (34.285714285714285em, -26.666666666666668em) {$X'$} ; 
\node (17) at (41.904761904761905em, -19.047619047619047em) {$A$} ; 
\node (18) at (41.904761904761905em, -26.666666666666668em) {$A'$} ; 
\node (19) at (49.523809523809526em, -26.666666666666668em) {$B'$} ; 
\node (20) at (30.380975882212322em, -22.791470482235862em) {$=$} ; 
\node (21) at (41.904761904761905em, -22.76573217482794em) {$A'$} ; 
\node (22) at (26.666666666666668em, -19.047619047619047em) {$A'$} ; 
\node (23) at (26.666666666666668em, -26.666666666666668em) {$B'$} ; 
\path 
(0) to[fore, black,->, curve={ratio=0.30000000000000004}, ] node[coordinate](24){} (1) 
(0) to[fore, black,->, curve={ratio=-0.2}, ] node[coordinate](25){} (1) 
(0) to[fore, black,->, ] node[coordinate](26){} (2) 
(1) to[fore, black,->, ] node[coordinate](27){} (3) 
(2) to[fore, black,->, curve={ratio=-0.1}, ] node[coordinate](28){} (3) 
(5) to[fore, black,->, curve={ratio=0.2}, ] node[coordinate](29){} (6) 
(5) to[fore, black,->, ] node[coordinate](30){} (7) 
(6) to[fore, black,->, ] node[coordinate](31){} (8) 
(7) to[fore, black,->, curve={ratio=0.2}, ] node[coordinate](32){} (8) 
(7) to[fore, black,->, curve={ratio=-0.2}, ] node[coordinate](33){} (8) 
(9) to[fore, black,->, curve={ratio=0.20000000000000004}, ] node[coordinate](34){} (10) 
(9) to[fore, black,->, curve={ratio=-0.10000000000000003}, ] node[coordinate](35){} (10) 
(9) to[fore, black,->, ] node[coordinate](36){} (11) 
(9) to[fore, black,->, ] node[coordinate](37){} (12) 
(12) to[fore, black,->, ] node[coordinate](38){} (13) 
(11) to[fore, black,->, ] node[coordinate](39){} (13) 
(12) to[fore, black,->, cell=0.2, ] node[coordinate](40){} (11) 
(10) to[fore, black,->, ] node[coordinate](41){} (14) 
(11) to[fore, black,->, ] node[coordinate](42){} (14) 
(14) to[fore, black,->, ] node[coordinate](43){} (23) 
(15) to[fore, black,->, curve={ratio=0.30000000000000004}, ] node[coordinate](44){} (16) 
(15) to[fore, black,->, ] node[coordinate](45){} (17) 
(16) to[fore, black,->, ] node[coordinate](46){} (18) 
(18) to[fore, black,->, ] node[coordinate](47){} (19) 
(16) to[fore, black,->, ] node[coordinate](48){} (21) 
(21) to[fore, black,->, ] node[coordinate](49){} (19) 
(21) to[fore, black,->, cell=0.2, ] node[coordinate](50){} (18) 
(17) to[fore, black,->, ] node[coordinate](51){} (21) 
(12) to[fore, black,->, ] node[coordinate](52){} (22) 
(22) to[fore, black,->, ] node[coordinate](53){} (23) 
(13) to[fore, black,->, ] node[coordinate](54){} (23) 
(24) to[fore, black,->, cell=0.2, ] node[coordinate](55){} (25) 
(32) to[fore, black,->, cell=0.2, ] node[coordinate](56){} (33) 
(34) to[fore, black,->, cell=0.2, ] node[coordinate](57){} (35) 
(52) to[fore, black,->, cell=0.2, ] node[coordinate](58){} (54) 
; 
\path[->] 
(0) edge["${\scriptstyle H_X}$"', pos=0.5, fore, black,->, curve={ratio=0.30000000000000004}, ] (1) 
(0) edge["${\scriptstyle H_X'}$", pos=0.5, fore, black,->, curve={ratio=-0.2}, ] (1) 
(0) edge["${\scriptstyle x}$", pos=0.5, fore, black,->, ] (2) 
(1) edge["${\scriptstyle x'}$"', pos=0.5, fore, black,->, ] (3) 
(2) edge["${\scriptstyle H_A'}$", pos=0.5, fore, black,->, curve={ratio=-0.1}, ] (3) 
(5) edge["${\scriptstyle H_X}$"', pos=0.5, fore, black,->, curve={ratio=0.2}, ] (6) 
(5) edge["${\scriptstyle x}$", pos=0.5, fore, black,->, ] (7) 
(6) edge["${\scriptstyle x'}$"', pos=0.5, fore, black,->, ] (8) 
(7) edge["${\scriptstyle H_A}$"', pos=0.5, fore, black,->, curve={ratio=0.2}, ] (8) 
(7) edge["${\scriptstyle H_A'}$", pos=0.5, fore, black,->, curve={ratio=-0.2}, ] (8) 
(9) edge["${\scriptstyle H_X}$"', pos=0.5, fore, black,->, curve={ratio=0.20000000000000004}, ] (10) 
(9) edge["${\scriptstyle H_X'}$", pos=0.5, fore, black,->, curve={ratio=-0.10000000000000003}, ] (10) 
(9) edge["${\scriptstyle x}$"', pos=0.5, fore, black,->, ] (11) 
(9) edge["${\scriptstyle x}$", pos=0.5, fore, black,->, ] (12) 
(12) edge["${\scriptstyle F}$", pos=0.5, fore, black,->, ] (13) 
(11) edge["${\scriptstyle G}$"', pos=0.5, fore, black,->, ] (13) 
(12) edge["${\scriptstyle \alpha}$", pos=0.5, fore, black,->, cell=0.2, ] (11) 
(10) edge["${\scriptstyle x'}$"', pos=0.5, fore, black,->, ] (14) 
(11) edge["${\scriptstyle H_A}$"', pos=0.5, fore, black,->, ] (14) 
(14) edge["${\scriptstyle G'}$"', pos=0.5, fore, black,->, ] (23) 
(15) edge["${\scriptstyle H_X}$"', pos=0.5, fore, black,->, curve={ratio=0.30000000000000004}, ] (16) 
(15) edge["${\scriptstyle x}$", pos=0.5, fore, black,->, ] (17) 
(16) edge["${\scriptstyle x'}$"', pos=0.5, fore, black,->, ] (18) 
(18) edge["${\scriptstyle G'}$"', pos=0.5, fore, black,->, ] (19) 
(16) edge["${\scriptstyle x'}$", pos=0.5, fore, black,->, ] (21) 
(21) edge["${\scriptstyle F'}$", pos=0.5, fore, black,->, ] (19) 
(21) edge["${\scriptstyle \alpha'}$", pos=0.5, fore, black,->, cell=0.2, ] (18) 
(17) edge["${\scriptstyle H_A}$", pos=0.5, fore, black,->, ] (21) 
(12) edge["${\scriptstyle H_A}$", pos=0.5, fore, black,->, ] (22) 
(22) edge["${\scriptstyle F'}$", pos=0.5, fore, black,->, ] (23) 
(13) edge["${\scriptstyle H_B}$"', pos=0.5, fore, black,->, ] (23) 
(24) edge["${\scriptstyle h_X}$", pos=0.5, fore, black,->, cell=0.2, ] (25) 
(32) edge["${\scriptstyle h_A}$", pos=0.5, fore, black,->, cell=0.2, ] (33) 
(34) edge["${\scriptstyle h_X}$", pos=0.5, fore, black,->, cell=0.2, ] (35) 
(52) edge["${\scriptstyle h_F}$", pos=0.5, fore, black,->, cell=0.2, ] (54) 
; 
\end{tikzpicture}
% END OF GENERATED LATEX
\]
\end{definition}
\cite{bird1989flexible}
%%% Local Variables:
%%% TeX-master: "../main.tex"
%%% End:
