\section{Application: Translation across simply-typed systems}

%%% The limitation of the recursion principle
Initial semantics represent higher-order languages as initial models over a
base category that depends on the type system of the language considered.
For instance, a simply-typed language can be represented as an initial model
over the category $[\Set^T,\Set^T]$, where $T$ is the initial $S$-algebra
representing the typing systems.
This is perfectly sensible, yet it limits the recursion principle provided
by initiality, as by definition it is confined to model over the same base
category.
It does not directly enable us to relate two languages with different type
systems $T$ and $T'$ as their models live above different categories
$[\Set^T,\Set^T]$ and $[\Set^{T'},\Set^{T'}]$.

This issue is not surprising as a translation of simply-typed languages
should first rely on a translation of typing systems $g : T \to T'$,
which is not accounted by initiality.
Actually, such a translation $g : T \to T'$ naturally acts on the underlying
categories:

\begin{proposition}
  \label{prop:adj-type}
  Given a morphism $g : T \to T'$, as both $T$ and $T'$ are small, $g$ admits left Kan extensions
  $\vec{g} := \Lan_g(\_) : \Set^T \to \Set^{T'}$
  defined by $\vec{g}(\Gamma)(t') := \bigsqcup_{t : T,\; g(t) = t'} \Gamma(t)$
  and denoted the retyping functor.
  Moreover, it is part of an adjunction between $\Set^T$ and $\Set^{T'}$,
  which in turn induces an adjunction on the endofunctors categories such
  that the functor $\overline{g} : [\Set^T,\Set^T] \to [\Set^{T'},\Set^{T'}]$
  is a monoidal.
  %
  \begin{align*}
    \begin{tikzcd}[ampersand replacement = \&, column sep=large]
      \Set^T \ar[r, bend left, "\vec{g}"] \ar[r, phantom, "\perp"]
        \& \Set^{T'} \ar[l, bend left, "g^*"]
    \end{tikzcd}
    &&
    \begin{tikzcd}[ampersand replacement = \&, column sep=large]
      {[}\Set^T{,}\Set^T{]} \ar[r, bend left, "\underline{g}(X_T) \;:=\; \vec{g} \circ X_T \circ g^*"]
                \ar[r, phantom, "\perp"]
        \& {[}\Set^{T'}{,}\Set^{T'}{]} \ar[l, bend left, "\overline{g}(X_{T'}) \,:=\, g^* \circ X_{T'} \circ \vec{g}"]
    \end{tikzcd}
  \end{align*}
\end{proposition}

% headline
A first solution to this issue was investigated by Ahrens \cite{ExtendedInitiality12}
by integrated the type system directly into the framework.
% How to
Throughout the framework, Ahrens replaced the category of monads over the
initial $S$-algebra $\Mon([\Set^{T},\Set^{T}])$ by the total category of
monads over the $S$-algebras\footnote{
  Actually, Ahrens used a category directly equivalent to this total
  category called $S$-monads.}:
%
\[\int_{K : S-\mathrm{Alg}} \Mon([\Set^K,\Set^K]) \]
%
% Objects
The objects of this category are pairs $(K,M)$ where $K$ is a $S$-algebra
representing a type system admitting a $S$-structure, and $M$ a monad over
$[\Set^{K},\Set^{K}]$ representing terms over $K$.
This accounts for much more languages as we are no longer restricted to
languages with type system $T$, we are allowed any languages with a type
system admitting a $S$-structure.
% Morphism
Morphisms $(K,M) \to (K',M')$ are then pairs $(g,f)$ where $g : K \to K'$ is
a morphism of $S$-algebra translating the type systems, and $f : M \to \overline{g}M'$
a morphism of monads\footnote{
  By \cref{prop:adj-type}, the functor $\overline{g} : [\Set^K,\Set^K] \to [\Set^{K'},\Set^{K'}]$
  associated to $g$ is monoidal, and hence generates a functor
  $\overline{g} : \Mon([\Set^K,\Set^K]) \to \Mon([\Set^{K'},\Set^{K'}])$.
}
translating terms after retying.
% Why it works
Using the total category then provides us a stronger recursion principle as
there is automatically a unique translation of languages from the initial model
to any other langages with a $\Sigma$-structure and a type system admitting a $S$-structure.
Moreover, unfolding the definition provides exactly the translation one would expect.
% Downsides
While this approach technically works, this is at the cost of technical modifications
to the entire framework in particular to signatures.
Furthermore, the endofunctors categories $[\Set^K,\Set^K]$ are now fully hardcoded
in the framework, whereas before it could take any fitting monoidal category as input
enabling to represent different kind of languages by changing the input category.

\todot{Attempted by me and Benedikt} \\
An attempted has been presented in \cite{IntroductionIS24} that is low level
but fails to provide a general result.

\todot{Us}
We pursue this approach and show that the metatheory presented in
\todot{cref} can be leverage to entend the recursion principle to
translating simply-type languages with different type systems, and that is
enables us to recover a stronger result then \cite{ExtendedInitiality12}.

\subsection{Small Functors}
\todoLipsum{1}

\subsection{Simply-Typed Algebraic as Small Functors}
\todoLipsum{1}

\subsection{Translation across simply-typed systems}
\todoLipsum{1}

\subsection{Initiality and Total Categories}
\todoLipsum{1}