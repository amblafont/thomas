\section{Application: Translation across simply-typed systems}

% The limitation of the recursion principle
Initial semantics represent higher-order languages
% with their substitution structure
as initial models over a base category that depends on the type system of
the language considered.
For instance, a simply-typed language with a set of types $T$ can be
represented as an initial model over the category $[\Set^T,\Set^T]$.
This is perfectly sensible, yet it limits the recursion principle provided
by initiality, as by definition it is confined to model over the same base
category.
Consequently, the recursion principle provided by initiality do
not directly enable us to relate two languages with different type
systems $T$ and $T'$ as their models live above different categories
$[\Set^T,\Set^T]$ and $[\Set^{T'},\Set^{T'}]$.

\vspace{3cm}

% Attempted by Benedikt
There has been attempted to go around this issue by Ahrens \cite{ExtendedInitiality12}.
However, to do so he had to modify the entire framework, and in particular
the definition of models.
Moreover, his presentation is specific to simply-typed languages, and it is
unclear whether it can be extended to more advanced languages.


% Attempted by me and Benedikt
An attempted has been presented in \cite{IntroductionIS24} that is low level.

We pursue this approach and show that the metatheory presented in
\todot{cref} can be leverage to entend the recursion principle to
translating languages with different type systems, and that is enables us to
recover a stronger result then \cite{ExtendedInitiality12}.


\subsection{pqst}
Asume given a translation of typing systems $T \to T'$.
% Outline
To translate monoids from $[\Set^{T'},\Set^{T'}]$ to $[\Set^T,\Set^T]$,
we start by translating contexts and endofunctors.
% LKE + g
$g$ admits a global left Kan extension, denoted
$\vec{g} := \Lan{g}{\_}$, that computes as below.
It entails an adjunction on contexts, i.e. between $\Set^T$ and $\Set^{T'}$:
%
\begin{align*}
  \begin{tikzcd}[ampersand replacement = \&, column sep=large]
      \Set^T
          \ar[r, bend left, "\vec{g}"]
          \ar[r, phantom, "\perp"]
        \& \Set^{T'} \ar[l, bend left, "g^*"]
  \end{tikzcd}
  &&
  \vec{g}(\Gamma)(t') \;:= \bigsqcup_{\substack{t : T \\ g(t) = t'}} \Gamma(t)
\end{align*}
%
This adjunction can be lifted to the endofunctors categories:

\begin{proposition}
  Given an adjunction $A \to B : k \vdash l : B \to A$ with unit $\eta : \Id
  \to l \circ k$ and counit $\epsilon : k \circ l \to \Id$, there is an
  adjunction on endofunctor categories:
  \begin{align*}
    \begin{tikzcd}[ampersand replacement = \&, column sep=large]
      A \ar[r, bend left, "k"]
                \ar[r, phantom, "\perp"]
        \& B \ar[l, bend left, "l"]
    \end{tikzcd}
    &&
    \begin{tikzcd}[ampersand replacement = \&, column sep=large]
        {[}A{,}A{]} \ar[r, bend left, "X_A \;\mapsto\; k \circ X_A \circ l"]
                  \ar[r, phantom, "\perp"]
          \& {[}B{,}B{]} \ar[l, bend left, "X_B \;\mapsto\; l \circ X_B \circ k"]
    \end{tikzcd}
  \end{align*}
  Moreover, the functor $G := X_B \;\mapsto\; l \circ X_B \circ k$ is
  \hyperref[def:monoidal-functor]{monoidal}, with $G_0 : \Id \to l \circ k
  := \eta$ and $G_2 : l \circ X_B \circ k \circ l \circ X_B' \circ k \to l
  \circ X_B \circ X_B' \circ k := l \circ X_B \circ \epsilon \circ X_B'\circ
  k$.
\end{proposition}

% Monoidal functor
\noindent Importantly, the adjunction on endofunctors categories provides
us with a monoidal functor $G : [\Set^{T'},\Set^{T'}] \to [\Set^T,\Set^T]$
that enables us to translate monoids:
% On monoids

\begin{proposition}
  \label{prop:functorial-monoids}
  Every monoidal functor $(F,F_0,F_2) : (\mc{C},I,\otimes) \to
  (\mc{D},J,\bullet)$ entails a functor on monoids $\Mon(\mc{C}) \to
  \Mon(\mc{D})$.
  It associates to any monoid $(R,\eta,\mu) : \Mon(\mc{C})$ the monoid
  $(F(R),F(\eta) \circ F_0,F(\mu) \circ F_2) : \Mon(\mc{D})$, and to any
  morphism of monoid $f : R \to R'$, the morphism $F(f) : F(R) \to F(R')$.
\end{proposition}

Hence, by theorem, to translate between this and this, it suffices to exhib this...



\subsection{Small Functors}
\todoLipsum{1}

\subsection{Simply-Typed Algebraic as Small Functors}
\todoLipsum{1}

\subsection{Translation across simply-typed systems}
\todoLipsum{1}

\subsection{Initiality and Total Categories}
\todoLipsum{1}