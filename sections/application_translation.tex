\section{Application: Translation across simply-typed systems}

%%% The limitation of the recursion principle
Initial semantics represent higher-order languages as initial models over a
base category that depends on the type system of the language considered.
For instance, a simply-typed language can be represented as an initial model
over the category $[\Set^T,\Set^T]$, where $T$ is the initial $S$-algebra
representing the typing systems.
This is perfectly sensible, yet it limits the recursion principle provided
by initiality, as by definition it is confined to model over the same base
category.
It does not directly enable us to relate two languages with different type
systems $T$ and $T'$ as their models live above different categories
$[\Set^T,\Set^T]$ and $[\Set^{T'},\Set^{T'}]$.

This issue is not surprising as a translation of simply-typed languages
should first rely on a translation of typing systems $g : T → T'$,
which is not accounted by initiality.
In fact, such a translation $g : T → T'$ naturally acts on the underlying
categories:
%
\begin{proposition}
  \label{prop:adj-type}
  Given a morphism $g : T → T'$, as both $T$ and $T'$ are small, $g$ admits left Kan extensions
  $\vec{g} := \Lan_g(\_) : \Set^T → \Set^{T'}$
  defined by $\vec{g}(Γ)(t') := \bigsqcup_{t : T,\; g(t) = t'} Γ(t)$
  and denoted the retyping functor.
  Moreover, it is part of an adjunction between $\Set^T$ and $\Set^{T'}$,
  that in turn, induces an adjunction on the endofunctors categories such
  that the functor $\ol{g} : [\Set^{T'},\Set^{T'}] → [\Set^T,\Set^T]$
  is a monoidal:
  %
  \begin{align*}
    \begin{tikzcd}[ampersand replacement = \&, column sep=large]
      \Set^T \ar[r, bend left, "\vec{g}"] \ar[r, phantom, "\perp"]
        \& \Set^{T'} \ar[l, bend left, "g^*"]
    \end{tikzcd}
    &&
    \begin{tikzcd}[ampersand replacement = \&, column sep=large]
      {[}\Set^T{,}\Set^T{]} \ar[r, bend left, "\underline{g}(X_T) \;:=\; \vec{g} ∘ X_T ∘ g^*"]
                \ar[r, phantom, "\perp"]
        \& {[}\Set^{T'}{,}\Set^{T'}{]} \ar[l, bend left, "\ol{g}(X_{T'}) \,:=\, g^* ∘ X_{T'} ∘ \vec{g}"]
    \end{tikzcd}
  \end{align*}
\end{proposition}
%
\noindent Moreover, as the functor $\ol{g} : [\Set^{K'},\Set^{K'}] →
[\Set^K,\Set^K]$ is monoidal, it induces a functor on the categories of monoids
$\ol{g} : \Mon([\Set^{K'},\Set^{K'}]) → \Mon([\Set^K,\Set^K])$,
monoids which are the base of the definition of models.

% headline
A first solution to this issue was proposed by Ahrens \cite{ExtendedInitiality12},
that leverage this fact to integrate the type system directly into the framework
in order to get a stronger recursion principle.
% How to
Throughout the framework, Ahrens replaced the category of monads over the
initial $S$-algebra $\Mon([\Set^{T},\Set^{T}])$ by the total category of
monads over the $S$-algebras\footnote{
  Actually, Ahrens used a category directly equivalent to this total
  category called $S$-monads.}:
%
\[\int_{K : S-\mathrm{Alg}} \Mon([\Set^K,\Set^K]) \]
%
% Objects
The objects of this category are pairs $(K,M)$ where $K$ is a $S$-algebra
representing a type system admitting a $S$-structure, and $M$ a monad over
$[\Set^{K},\Set^{K}]$ representing terms over $K$.
This accounts for much more languages as we are no longer restricted to
languages with type system $T$, we are allowed any languages with a type
system admitting a $S$-structure.
% Morphism
Morphisms $(K,M) → (K',M')$ are then pairs $(g,f)$ where $g : K → K'$ is
a morphism of $S$-algebra translating the type systems, and $f : M → \ol{g}M'$
a morphism of monads translating terms after retying.
Note, that $\ol{g}$ is essential here for homogeneity.
% Why it works
Using the total category then provides us a stronger recursion principle as
there is automatically a unique translation of languages from the initial model
to any other langages with a type system admitting a $S$-structure, and constructors admitting a $\Sigma$-structure.
Moreover, unfolding the definition provides us exactly the usual translation one would expect.
% Downsides
While this approach technically works, it is internal to the framework, and
is at the cost of technical modifications to the entirety of it, in
particular to the notion of signatures.
Furthermore, the endofunctors categories $[\Set^K,\Set^K]$ are now fully hardcoded
in the framework, whereas before it could take any fitting monoidal category as input
enabling to represent different kind of languages by changing the input category.

% Intro
To go around this issue, rather than modifying the framework to get a
stronger recursion principle, Lamiaux and Ahrens \cite[Section 9.4]{IntroductionIS24}
suggested an external solution that manipulates the framework with modifying it.
% Method
To translate a language to another, they suggested to directly translate the
monoid underlying the model of the target language using the monoidal
functor $\ol{g}$, and to then build a model of the source language on
top of it to get a translation by initiality.
% Example
They investigated this idea through the example of type erasure from the
simply-typed lambda calculus to the untyped one, and shown that once the
definition unfolded, it is easy to build a model on top of the translated
monoid provided tedious bureaucracy.
%
While this approach do not have the downsides of the internal approach and
seems to work on well chosen examples, it remains to provide a general
procedure.

% Introduction
To make the external approach viable, it is necessary to identify general
conditions for it to apply, when those conditions are satisfied, and in
which case if they are reasonable enough to be applied in practice.
% theorem
As $\ol{g} : [\Set^{T'},\Set^{T'}] → [\Set^T,\Set^T]$ is monoidal, applying
the metatheoretical result \cref{thm:fct-models} provide us with such
general conditions.
%
Given two signatures  $Σ_{T} : \Sig([\Set^T,\Set^T])$ and $Σ_{T'} : \Sig([\Set^{'T'},\Set^{'T'}])$
it suffices for natural transformation $α : Σ_{T} \ol{g} → \ol{g} Σ_{T'}$ such
that $π_1 \circ α = \Id$, for it to exist a functor:
%
\[ \ol{g} : \Model(Σ_{T'}) → \Model(Σ_{T}) \]
%
In the following, we that that every binding signature $\Sigma_{T}$ and
morphism of $S$-algebra $g : T \to T'$ induces a signature $\ol{\Sigma_{T}}$
that satisfies the condition, and hence a functor $\ol{g} : \Model(\ol{Σ_{T}}) → \Model(Σ_{T})$
that enables to translate languages.
%
We further show that it enables to build a total category externally to the
framework, and to prove general initiality theorem to the framework,
recovering a more general result than the internal approach without its
downsides.

\subsection{Small Functors and Algebraic Signatures}
\todot{
\begin{itemize}
  \item Definition Small functor
  \item Alg sig as small functor
  \item Small fct -> Sig strength
\end{itemize}}

\subsection{Translation across simply-typed systems}
\todot{
\begin{itemize}
  \item Traduction sig as small fact
  \item Proof theorem
\end{itemize}}

\subsection{Initiality and Total Categories}
\todot{
\begin{itemize}
  \item Total category
  \item Initiality result
  \item Traduction + Example
\end{itemize}}
