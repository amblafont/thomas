\section{Inserters}

\todot{Write a plan}
\begin{definition}
  An \emph{inserter diagram $(F,G)$} in a 2-category $ℬ$ is a pair of parallel 1-cells
  $
  \begin{tikzcd}
  A \doublerightarrow{r}{F}{G} & B
  \end{tikzcd}
  $.
  A \emph{cone} of such an inserter diagram is a pair $(X\xrightarrow{x} A,α)$, sometimes abbreviated as $\alpha$, consisting of a
  1-cell $X\xrightarrow{x} A$ and a 2-cell $ F · x \xrightarrow{α} G · x$.
  A \emph{morphism} between two cones $(X\xrightarrow{x}A,α)$ and $(X\xrightarrow{x'}A,α')$ is
  a 2-cell $x \xrightarrow{γ} x'$ such that the two induced 2-cells from $F∘x$ to
  $G∘x'$ are equal as in Equation~\eqref{eq:mor-cones}.
  \begin{equation}
    \label{eq:mor-cones}
    % YADE DIAGRAM diagrams/ins-2universal-prop.json
% GENERATED LATEX
\begin{tikzpicture}[every node/.style={inner sep=2pt,outer sep=0pt,anchor=base,text height=1.2ex, text depth=0.25ex}] 
\node (0) at (2.2857142857142856em, -6.857142857142857em) {$X$} ; 
\node (1) at (11.428571428571429em, -6.857142857142857em) {$B$} ; 
\node (2) at (6.857142857142857em, -6.857142857142857em) {$A$} ; 
\node (3) at (6.857142857142857em, -11.428571428571429em) {$A$} ; 
\node (4) at (13.680774400214188em, -8.458880769104516em) {$=$} ; 
\node (5) at (16em, -6.857142857142857em) {$X$} ; 
\node (6) at (23.684848804192em, -6.839611636006773em) {$B$} ; 
\node (7) at (20.571428571428573em, -6.857142857142857em) {$A$} ; 
\node (8) at (20.571428571428573em, -11.428571428571429em) {$A$} ; 
\path 
(0) to[fore, black,->, bend right={-40.10704565915762}, ] node[coordinate](9){} (2) 
(2) to[fore, black,->, ] node[coordinate](10){} (1) 
(0) to[fore, black,->, bend right={34.37746770784939}, ] node[coordinate](11){} (2) 
(0) to[fore, black,->, bend right={17.1887338539247}, ] node[coordinate](12){} (3) 
(3) to[fore, black,->, bend right={11.459155902616464}, ] node[coordinate](13){} (1) 
(2) to[fore, black,->, cell=0.2, ] node[coordinate](14){} (3) 
(5) to[fore, black,->, ] node[coordinate](15){} (7) 
(7) to[fore, black,->, ] node[coordinate](16){} (6) 
(5) to[fore, black,->, bend right={-17.1887338539247}, ] node[coordinate](17){} (8) 
(8) to[fore, black,->, bend right={5.729577951308232}, ] node[coordinate](18){} (6) 
(7) to[fore, black,->, cell=0.2, ] node[coordinate](19){} (8) 
(5) to[fore, black,->, bend right={40.10704565915762}, ] node[coordinate](20){} (8) 
(9) to[fore, black,->, cell=0.2, ] node[coordinate](21){} (11) 
(17) to[fore, black,->, cell=0.2, ] node[coordinate](22){} (20) 
; 
\path[->] 
(0) edge["${\scriptstyle x}$", pos=0.5, fore, black,->, bend right={-40.10704565915762}, ] (2) 
(2) edge["${\scriptstyle F}$", pos=0.5, fore, black,->, ] (1) 
(0) edge["${\scriptstyle x'}$"', pos=0.5, fore, black,->, bend right={34.37746770784939}, ] (2) 
(0) edge["${\scriptstyle x'}$"', pos=0.5, fore, black,->, bend right={17.1887338539247}, ] (3) 
(3) edge["${\scriptstyle G}$"', pos=0.5, fore, black,->, bend right={11.459155902616464}, ] (1) 
(2) edge["${\scriptstyle α'}$", pos=0.5, fore, black,->, cell=0.2, ] (3) 
(5) edge["${\scriptstyle x}$", pos=0.5, fore, black,->, ] (7) 
(7) edge["${\scriptstyle F}$", pos=0.5, fore, black,->, ] (6) 
(5) edge["${\scriptstyle x}$", pos=0.5, fore, black,->, bend right={-17.1887338539247}, ] (8) 
(8) edge["${\scriptstyle G}$"', pos=0.5, fore, black,->, bend right={5.729577951308232}, ] (6) 
(7) edge["${\scriptstyle α}$", pos=0.5, fore, black,->, cell=0.2, ] (8) 
(5) edge["${\scriptstyle x'}$"', pos=0.5, fore, black,->, bend right={40.10704565915762}, ] (8) 
(9) edge["${\scriptstyle γ}$", pos=0.5, fore, black,->, cell=0.2, ] (11) 
(17) edge["${\scriptstyle γ}$", pos=0.5, fore, black,->, cell=0.2, ] (20) 
; 
\end{tikzpicture}
% END OF GENERATED LATEX
  \end{equation}
\end{definition}
\begin{remark}
  \label{rem:cons-cone}
  Cones are stable by precomposition in the following sense:
  if $(X\xrightarrow{x} A,α)$ is a cone, then
  \begin{itemize}
  \item for any 1-cell $x' \xrightarrow{f} x$, 
  the 2-cell $F ⋅ x' \xrightarrow{\alpha ⋅ f} G ⋅ x'$ defines a cone;
  \item for any 2-cell $h$ between $f,f'∶ X' → X$, 
  the 2-cell $\alpha ⋅ h$ is a cone morphism 
  between $\alpha ⋅ f$ and $\alpha ⋅ f'$. 
  \end{itemize}
\end{remark}
  % as in the following diagram:
%   \[
%   OLD YADE DIAGRAM diagrams/inserter.json
% % GENERATED LATEX
% \begin{tikzpicture}[every node/.style={inner sep=5pt,outer sep=0pt,anchor=base,text height=1.2ex, text depth=0.25ex}] 
\node (0) at (7.142857142857143em, -4.285714285714286em) {$X$} ; 
\node (1) at (10em, -1.4285714285714286em) {$A$} ; 
\node (2) at (12.857142857142858em, -4.285714285714286em) {$B$} ; 
\node (3) at (10em, -7.142857142857143em) {$A$} ; 
\path 
(0) to[fore, black,->, ] node[coordinate](4){} (1) 
(1) to[fore, black,->, ] node[coordinate](5){} (2) 
(0) to[fore, black,->, ] node[coordinate](6){} (3) 
(3) to[fore, black,->, ] node[coordinate](7){} (2) 
(1) to[fore, black,->, cell=0.2, ] node[coordinate](8){} (3) 
; 
\path[->] 
(0) edge["${\scriptstyle x}$", pos=0.5, fore, black,->, ] (1) 
(1) edge["${\scriptstyle F}$", pos=0.5, fore, black,->, ] (2) 
(0) edge["${\scriptstyle x}$"', pos=0.5, fore, black,->, ] (3) 
(3) edge["${\scriptstyle G}$"', pos=0.5, fore, black,->, ] (2) 
(1) edge["${\scriptstyle α}$", pos=0.5, fore, black,->, cell=0.2, ] (3) 
; 
\end{tikzpicture}
% % END OF GENERATED LATEX
% \]
\begin{definition}
A limit of an inserter diagram $(F,G)$, called the \emph{inserter of $F$ and
  $G$}, is a cone $(I\xrightarrow{i}A, κ)$ that is universal, in the following sense.
  \begin{itemize}
    \item 
  Any cone $ F · x \xrightarrow{α} G · x$  is of the shape 
  $κ ⋅ \universalins{\alpha}$ for a unique 
  $\universalins{\alpha}$, as summarised in the diagram below.
  \[
% YADE DIAGRAM diagrams/ins-universal-prop.json
% GENERATED LATEX
\begin{tikzpicture}[every node/.style={inner sep=2pt,outer sep=0pt,anchor=base,text height=1.2ex, text depth=0.25ex}] 
\node (0) at (18.333333333333332em, -8.333333333333334em) {$X$} ; 
\node (1) at (21.666666666666668em, -5em) {$A$} ; 
\node (2) at (25em, -8.333333333333334em) {$B$} ; 
\node (3) at (21.666666666666668em, -11.666666666666666em) {$A$} ; 
\node (4) at (28.333333333333332em, -8.333333333333334em) {$=$} ; 
\node (5) at (31.666666666666668em, -8.333333333333334em) {$X$} ; 
\node (6) at (38.333333333333336em, -5em) {$A$} ; 
\node (7) at (41.666666666666664em, -8.333333333333334em) {$B$} ; 
\node (8) at (38.333333333333336em, -11.666666666666666em) {$A$} ; 
\node (9) at (35em, -8.333333333333334em) {$I$} ; 
\path 
(0) to[fore, black,->, ] node[coordinate](10){} (1) 
(1) to[fore, black,->, ] node[coordinate](11){} (2) 
(0) to[fore, black,->, ] node[coordinate](12){} (3) 
(3) to[fore, black,->, ] node[coordinate](13){} (2) 
(1) to[fore, black,->, cell=0.2, ] node[coordinate](14){} (3) 
(6) to[fore, black,->, ] node[coordinate](15){} (7) 
(8) to[fore, black,->, ] node[coordinate](16){} (7) 
(6) to[fore, black,->, cell=0.2, ] node[coordinate](17){} (8) 
(5) to[fore, black,->, dashed, ] node[coordinate](18){} (9) 
(9) to[fore, black,->, ] node[coordinate](19){} (8) 
(5) to[fore, black,->, bend right={-17.1887338539247}, ] node[coordinate](20){} (6) 
(5) to[fore, black,->, bend right={17.1887338539247}, ] node[coordinate](21){} (8) 
(9) to[fore, black,->, ] node[coordinate](22){} (6) 
; 
\path[->] 
(0) edge["${\scriptstyle x}$", pos=0.5, fore, black,->, ] (1) 
(1) edge["${\scriptstyle F}$", pos=0.5, fore, black,->, ] (2) 
(0) edge["${\scriptstyle x}$"', pos=0.5, fore, black,->, ] (3) 
(3) edge["${\scriptstyle G}$"', pos=0.5, fore, black,->, ] (2) 
(1) edge["${\scriptstyle α}$", pos=0.5, fore, black,->, cell=0.2, ] (3) 
(6) edge["${\scriptstyle F}$", pos=0.5, fore, black,->, ] (7) 
(8) edge["${\scriptstyle G}$"', pos=0.5, fore, black,->, ] (7) 
(6) edge["${\scriptstyle κ}$", pos=0.5, fore, black,->, cell=0.2, ] (8) 
(5) edge["${\scriptstyle ∃!\universalins{α}}$", pos=0.7, fore, black,->, dashed, ] (9) 
(9) edge["${\scriptstyle i}$"', pos=0.30000000000000004, fore, black,->, ] (8) 
(5) edge["${\scriptstyle x}$", pos=0.5, fore, black,->, bend right={-17.1887338539247}, ] (6) 
(5) edge["${\scriptstyle x}$"', pos=0.5, fore, black,->, bend right={17.1887338539247}, ] (8) 
(9) edge["${\scriptstyle i}$", pos=0.5, fore, black,->, ] (6) 
; 
\end{tikzpicture}
% END OF GENERATED LATEX
\]
\item
  Any cone morphism
  $x \xrightarrow{γ} x'$
  between  $ F · x \xrightarrow{α} G · x$ and 
  $ F · x' \xrightarrow{α'} G · x'$ is of the shape 
  $\kappa ⋅ \universalins{\gamma}$ 
  for a unique $\universalins{\gamma}∶ \universalins{\alpha} → \universalins{\alpha'}$, as summarised in the diagram below.
\[ 
  % YADE DIAGRAM diagrams/ins-2universal-last.json
% GENERATED LATEX
\begin{tikzpicture}[every node/.style={inner sep=2pt,outer sep=0pt,anchor=base,text height=1.2ex, text depth=0.25ex}] 
\node (0) at (13.095238095238095em, -5.9523809523809526em) {$X$} ; 
\node (1) at (20.238095238095237em, -5.9523809523809526em) {$A$} ; 
\node (2) at (17.857142857142858em, -5.9523809523809526em) {$I$} ; 
\node (3) at (3.5714285714285716em, -5.9523809523809526em) {$X$} ; 
\node (4) at (8.333333333333334em, -5.9523809523809526em) {$A$} ; 
\node (5) at (10.714285714285714em, -5.9523809523809526em) {$=$} ; 
\path 
(0) to[fore, black,->, curve={ratio=-0.5}, ] node[coordinate](6){} (2) 
(0) to[fore, black,->, curve={ratio=-0.7999999999999999}, ] node[coordinate](7){} (1) 
(0) to[fore, black,->, curve={ratio=0.7999999999999999}, ] node[coordinate](8){} (1) 
(2) to[fore, black,->, ] node[coordinate](9){} (1) 
(0) to[fore, black,->, curve={ratio=0.4}, ] node[coordinate](10){} (2) 
(3) to[fore, black,->, curve={ratio=-0.5}, ] node[coordinate](11){} (4) 
(3) to[fore, black,->, curve={ratio=0.5}, ] node[coordinate](12){} (4) 
(6) to[fore, black,->, cell=0.2, dashed, ] node[coordinate](13){} (10) 
(11) to[fore, black,->, cell=0.2, ] node[coordinate](14){} (12) 
; 
\path[->] 
(0) edge["${\scriptstyle \universalins{α}}$", pos=0.7, fore, black,->, curve={ratio=-0.5}, ] (2) 
(0) edge["${\scriptstyle x}$", pos=0.5, fore, black,->, curve={ratio=-0.7999999999999999}, ] (1) 
(0) edge["${\scriptstyle x'}$"', pos=0.5, fore, black,->, curve={ratio=0.7999999999999999}, ] (1) 
(0) edge["${\scriptstyle \universalins{α'}}$"', pos=0.7999999999999999, fore, black,->, curve={ratio=0.4}, ] (2) 
(3) edge["${\scriptstyle x}$", pos=0.5, fore, black,->, curve={ratio=-0.5}, ] (4) 
(3) edge["${\scriptstyle x'}$"', pos=0.5, fore, black,->, curve={ratio=0.5}, ] (4) 
(6) edge["${\scriptstyle ∃!\universalins{γ}}$", pos=0.5, fore, black,->, cell=0.2, dashed, ] (10) 
(11) edge["${\scriptstyle \gamma}$", pos=0.5, fore, black,->, cell=0.2, ] (12) 
(2) edge["${\scriptstyle i}$", pos=0.5, fore, black,->, ] (1) 
; 
\end{tikzpicture}
% END OF GENERATED LATEX
  \]
  % , a
  % 2-cell  such that the two induced 2-cells from $F∘x$ to
  % $G∘x'$ are equal as in Equation~\eqref{eq:mor-cones}, 
  % then these 2-cells are also equal to $κ · \universalins{γ}$, as
  % in~\eqref{eq:mediating-gamma},
  % for a unique $\universalins{α} \xrightarrow{\universalins{γ}}\universalins{α'}$.

%   \begin{equation}
%     \label{eq:mediating-gamma}
%     % YADE DIAGRAM diagrams/ins-2universal-alpha.json
% % GENERATED LATEX
% \begin{tikzpicture}[every node/.style={inner sep=2pt,outer sep=0pt,anchor=base,text height=1.2ex, text depth=0.25ex}] 
\node (0) at (5.9523809523809526em, -8.333333333333334em) {$X$} ; 
\node (1) at (13.095238095238095em, -5.9523809523809526em) {$A$} ; 
\node (2) at (15.476190476190476em, -8.333333333333334em) {$B$} ; 
\node (3) at (13.095238095238095em, -10.714285714285714em) {$A$} ; 
\node (4) at (10.714285714285714em, -8.333333333333334em) {$I$} ; 
\path 
(1) to[fore, black,->, ] node[coordinate](5){} (2) 
(3) to[fore, black,->, ] node[coordinate](6){} (2) 
(1) to[fore, black,->, cell=0.2, ] node[coordinate](7){} (3) 
(0) to[fore, black,->, curve={ratio=-0.5}, ] node[coordinate](8){} (4) 
(4) to[fore, black,->, ] node[coordinate](9){} (3) 
(0) to[fore, black,->, curve={ratio=-0.4}, ] node[coordinate](10){} (1) 
(0) to[fore, black,->, curve={ratio=0.4}, ] node[coordinate](11){} (3) 
(4) to[fore, black,->, ] node[coordinate](12){} (1) 
(0) to[fore, black,->, curve={ratio=0.5}, ] node[coordinate](13){} (4) 
(8) to[fore, black,->, cell=0.2, dashed, ] node[coordinate](14){} (13) 
; 
\path[->] 
(1) edge["${\scriptstyle F}$", pos=0.5, fore, black,->, ] (2) 
(3) edge["${\scriptstyle G}$"', pos=0.5, fore, black,->, ] (2) 
(1) edge["${\scriptstyle κ}$", pos=0.5, fore, black,->, cell=0.2, ] (3) 
(0) edge["${\scriptstyle \universalins{α}}$", pos=0.7, fore, black,->, curve={ratio=-0.5}, ] (4) 
(4) edge["${\scriptstyle i}$"', pos=0.30000000000000004, fore, black,->, ] (3) 
(0) edge["${\scriptstyle x}$", pos=0.20000000000000004, fore, black,->, curve={ratio=-0.4}, ] (1) 
(0) edge["${\scriptstyle x'}$"', pos=0.20000000000000004, fore, black,->, curve={ratio=0.4}, ] (3) 
(0) edge["${\scriptstyle \universalins{α'}}$"', pos=0.7999999999999999, fore, black,->, curve={ratio=0.5}, ] (4) 
(8) edge["${\scriptstyle ∃!\universalins{γ}}$", pos=0.5, fore, black,->, cell=0.2, dashed, ] (13) 
(4) edge["${\scriptstyle i}$", pos=0.5, fore, black,->, ] (1) 
; 
\end{tikzpicture}
% % END OF GENERATED LATEX
%     \end{equation}
%   %\[
%     % YADE DIAGRAM diagrams/ins-2universal-all.json
% % GENERATED LATEX
% % \begin{tikzpicture}[every node/.style={inner sep=2pt,outer sep=0pt,anchor=base,text height=1.2ex, text depth=0.25ex}] 
\node (0) at (23.607876323518298em, -11.968874976748513em) {$X$} ; 
\node (1) at (30.750733466375443em, -9.587922595796131em) {$A$} ; 
\node (2) at (33.13168584732782em, -11.968874976748513em) {$B$} ; 
\node (3) at (30.750733466375443em, -14.349827357700892em) {$A$} ; 
\node (4) at (28.36978108542306em, -11.968874976748513em) {$I$} ; 
\node (5) at (21.56040196191697em, -11.901363554454985em) {$=$} ; 
\node (6) at (4.714430400303432em, -10.714285714285714em) {$X$} ; 
\node (7) at (11.190620876493908em, -10.714285714285714em) {$B$} ; 
\node (8) at (8.333333333333334em, -10.714285714285714em) {$A$} ; 
\node (9) at (8.333333333333334em, -15.476190476190476em) {$A$} ; 
\node (10) at (14.000144686017718em, -10.714285714285714em) {$X$} ; 
\node (11) at (20.095382781255815em, -10.653253101167225em) {$B$} ; 
\node (12) at (17.285858971732004em, -10.714285714285714em) {$A$} ; 
\node (13) at (17.285858971732004em, -15.476190476190476em) {$A$} ; 
\node (14) at (12.714430400303431em, -11.901363554454985em) {$=$} ; 
\path 
(1) to[fore, black,->, ] node[coordinate](15){} (2) 
(3) to[fore, black,->, ] node[coordinate](16){} (2) 
(1) to[fore, black,->, cell=0.2, ] node[coordinate](17){} (3) 
(0) to[fore, black,->, curve={ratio=-0.5}, ] node[coordinate](18){} (4) 
(4) to[fore, black,->, ] node[coordinate](19){} (3) 
(0) to[fore, black,->, curve={ratio=-0.4}, ] node[coordinate](20){} (1) 
(0) to[fore, black,->, curve={ratio=0.4}, ] node[coordinate](21){} (3) 
(4) to[fore, black,->, ] node[coordinate](22){} (1) 
(0) to[fore, black,->, curve={ratio=0.5}, ] node[coordinate](23){} (4) 
(6) to[fore, black,->, curve={ratio=-0.5}, ] node[coordinate](24){} (8) 
(8) to[fore, black,->, ] node[coordinate](25){} (7) 
(6) to[fore, black,->, curve={ratio=0.4}, ] node[coordinate](26){} (8) 
(6) to[fore, black,->, curve={ratio=0.30000000000000004}, ] node[coordinate](27){} (9) 
(9) to[fore, black,->, curve={ratio=0.2}, ] node[coordinate](28){} (7) 
(8) to[fore, black,->, cell=0.2, ] node[coordinate](29){} (9) 
(10) to[fore, black,->, ] node[coordinate](30){} (12) 
(12) to[fore, black,->, ] node[coordinate](31){} (11) 
(10) to[fore, black,->, curve={ratio=-0.1}, ] node[coordinate](32){} (13) 
(13) to[fore, black,->, curve={ratio=0.1}, ] node[coordinate](33){} (11) 
(12) to[fore, black,->, cell=0.2, ] node[coordinate](34){} (13) 
(10) to[fore, black,->, curve={ratio=0.6}, ] node[coordinate](35){} (13) 
(18) to[fore, black,->, cell=0.2, dashed, ] node[coordinate](36){} (23) 
(24) to[fore, black,->, cell=0.2, ] node[coordinate](37){} (26) 
(32) to[fore, black,->, cell=0.2, ] node[coordinate](38){} (35) 
; 
\path[->] 
(1) edge["${\scriptstyle F}$", pos=0.5, fore, black,->, ] (2) 
(3) edge["${\scriptstyle G}$"', pos=0.5, fore, black,->, ] (2) 
(1) edge["${\scriptstyle κ}$", pos=0.5, fore, black,->, cell=0.2, ] (3) 
(0) edge["${\scriptstyle \universalins{α}}$", pos=0.7, fore, black,->, curve={ratio=-0.5}, ] (4) 
(4) edge["${\scriptstyle i}$"', pos=0.30000000000000004, fore, black,->, ] (3) 
(0) edge["${\scriptstyle x}$", pos=0.5, fore, black,->, curve={ratio=-0.4}, ] (1) 
(0) edge["${\scriptstyle x'}$"', pos=0.5, fore, black,->, curve={ratio=0.4}, ] (3) 
(0) edge["${\scriptstyle \universalins{α'}}$"', pos=0.7999999999999999, fore, black,->, curve={ratio=0.5}, ] (4) 
(6) edge["${\scriptstyle x}$", pos=0.5, fore, black,->, curve={ratio=-0.5}, ] (8) 
(8) edge["${\scriptstyle F}$", pos=0.5, fore, black,->, ] (7) 
(6) edge["${\scriptstyle x'}$"', pos=0.5, fore, black,->, curve={ratio=0.4}, ] (8) 
(6) edge["${\scriptstyle x'}$"', pos=0.5, fore, black,->, curve={ratio=0.30000000000000004}, ] (9) 
(9) edge["${\scriptstyle G}$"', pos=0.5, fore, black,->, curve={ratio=0.2}, ] (7) 
(8) edge["${\scriptstyle α'}$", pos=0.5, fore, black,->, cell=0.2, ] (9) 
(10) edge["${\scriptstyle x}$", pos=0.5, fore, black,->, ] (12) 
(12) edge["${\scriptstyle F}$", pos=0.5, fore, black,->, ] (11) 
(10) edge["${\scriptstyle x}$", pos=0.30000000000000004, fore, black,->, curve={ratio=-0.1}, ] (13) 
(13) edge["${\scriptstyle G}$"', pos=0.5, fore, black,->, curve={ratio=0.1}, ] (11) 
(12) edge["${\scriptstyle α}$", pos=0.5, fore, black,->, cell=0.2, ] (13) 
(10) edge["${\scriptstyle x'}$"', pos=0.5, fore, black,->, curve={ratio=0.6}, ] (13) 
(18) edge["${\scriptstyle ∃!\universalins{γ}}$", pos=0.5, fore, black,->, cell=0.2, dashed, ] (23) 
(24) edge["${\scriptstyle γ}$", pos=0.5, fore, black,->, cell=0.2, ] (26) 
(32) edge["${\scriptstyle γ}$", pos=0.5, fore, black,->, cell=0.2, ] (35) 
(4) edge["${\scriptstyle i}$", pos=0.5, fore, black,->, ] (1) 
; 
\end{tikzpicture} 
% % END OF GENERATED LATEX
%   %\]
\end{itemize}
\end{definition}
% \begin{remark}
%   \label{rem:cons-cone}
%   Any 1-cell $X \xrightarrow{x} I$ induces a cone 
%   $(X \xrightarrow{x} I → A, x ⋅ κ)$ and any 2-cell
%   $     
%   % YADE DIAGRAM diagrams/a-2-cell.json
%   % GENERATED LATEX
%   \begin{tikzpicture}[every node/.style={inner sep=2pt,outer sep=0pt,anchor=base,text height=1.2ex, text depth=0.25ex}] 
\node (0) at (5.119047619047619em, -5.119047619047619em) {$X$} ; 
\node (1) at (9.3810970669701em, -5.119047619047619em) {$I$} ; 
\path 
(0) to[fore, black,->, curve={ratio=-0.4}, ] node[coordinate](2){} (1) 
(0) to[fore, black,->, curve={ratio=0.4}, ] node[coordinate](3){} (1) 
(2) to[fore, black,->, cell=0.2, ] node[coordinate](4){} (3) 
; 
\path[->] 
(0) edge["${\scriptstyle x}$", pos=0.5, fore, black,->, curve={ratio=-0.4}, ] (1) 
(0) edge["${\scriptstyle x'}$"', pos=0.5, fore, black,->, curve={ratio=0.4}, ] (1) 
(2) edge["${\scriptstyle γ}$", pos=0.5, fore, black,->, cell=0.2, ] (3) 
; 
\end{tikzpicture}
%   % END OF GENERATED LATEX
%   $ induces a cone morphism between 
%   $(X \xrightarrow{x} I → A, x ⋅ κ)$ and 
%   $(X \xrightarrow{x'} I → A, x' ⋅ κ)$.
%   The universal property says that actually all cones and cone morphisms are of this shape.
% \end{remark}
 \begin{definition}
   A \emph{morphism of inserter diagrams} between
   $
   \begin{tikzcd}
     A \doublerightarrow{r}{F}{G} & B
   \end{tikzcd}
   $
   and 
   $
   \begin{tikzcd}
     A' \doublerightarrow{r}{F'}{G'} & B'
   \end{tikzcd}
   $
   consists of 1-cells $A \xrightarrow{H_A} A'$, $B\xrightarrow{H_B} B'$, and
   2-cells $  F' H_A \xrightarrow{h_F}  H_BF$ and $H_B G \xrightarrow{h_G} G' H_A$
   as in the following diagram
   \[
% YADE DIAGRAM diagrams/morphism-inserter.json
% GENERATED LATEX
\begin{tikzpicture}[every node/.style={inner sep=2pt,outer sep=0pt,anchor=base,text height=1.2ex, text depth=0.25ex}] 
\node (0) at (11.428571428571429em, -3.8095238095238093em) {$A$} ; 
\node (1) at (19.047619047619047em, -3.8095238095238093em) {$B$} ; 
\node (2) at (19.047619047619047em, -11.428571428571429em) {$B'$} ; 
\node (3) at (11.428571428571429em, -11.428571428571429em) {$A'$} ; 
\path 
(0) to[fore, black,->, curve={ratio=-0.1}, ] node[coordinate](4){} (1) 
(0) to[fore, black,->, curve={ratio=0.1}, ] node[coordinate](5){} (1) 
(1) to[fore, black,->, ] node[coordinate](6){} (2) 
(0) to[fore, black,->, ] node[coordinate](7){} (3) 
(3) to[fore, black,->, curve={ratio=-0.1}, ] node[coordinate](8){} (2) 
(3) to[fore, black,->, curve={ratio=0.1}, ] node[coordinate](9){} (2) 
(7) to[fore, black,->, cell=0.2, curve={ratio=-0.2}, ] node[coordinate](10){} (6) 
(6) to[fore, black,->, cell=0.2, curve={ratio=-0.2}, ] node[coordinate](11){} (7) 
; 
\path[->] 
(0) edge["${\scriptstyle F}$", pos=0.5, fore, black,->, curve={ratio=-0.1}, ] (1) 
(0) edge["${\scriptstyle G}$"', pos=0.5, fore, black,->, curve={ratio=0.1}, ] (1) 
(1) edge["${\scriptstyle H_B}$", pos=0.5, fore, black,->, ] (2) 
(0) edge["${\scriptstyle H_A}$"', pos=0.5, fore, black,->, ] (3) 
(3) edge["${\scriptstyle F'}$", pos=0.5, fore, black,->, curve={ratio=-0.1}, ] (2) 
(3) edge["${\scriptstyle G'}$"', pos=0.5, fore, black,->, curve={ratio=0.1}, ] (2) 
(7) edge["${\scriptstyle h_F}$", pos=0.5, fore, black,->, cell=0.2, curve={ratio=-0.2}, ] (6) 
(6) edge["${\scriptstyle h_G}$", pos=0.5, fore, black,->, cell=0.2, curve={ratio=-0.2}, ] (7) 
; 
\end{tikzpicture}
% END OF GENERATED LATEX
     \]
   \end{definition}

      \begin{lemma}
   \label{lem:mor-inserter}
   Using the same notations as above, given a morphism of inserter diagrams
   there exists a unique
   1-cell $I\xrightarrow{H_I}I'$ between the inserters such that $H_A ∘ i = i' ∘
   H_I$, and the following equality holds.
  \[
% YADE DIAGRAM diagrams/morphism-inserter-univ.json
% GENERATED LATEX
\begin{tikzpicture}[every node/.style={inner sep=2pt,outer sep=0pt,anchor=base,text height=1.2ex, text depth=0.25ex}] 
\node (0) at (1.619047619047619em, -11.333333333333334em) {$I$} ; 
\node (1) at (4.857142857142857em, -8.095238095238095em) {$A$} ; 
\node (2) at (8.095238095238095em, -11.333333333333334em) {$B$} ; 
\node (3) at (4.857142857142857em, -14.571428571428571em) {$A$} ; 
\node (4) at (11.333333333333334em, -11.333333333333334em) {$B'$} ; 
\node (5) at (8.095238095238095em, -14.571428571428571em) {$A'$} ; 
\node (6) at (8.095238095238095em, -8.095238095238095em) {$A'$} ; 
\node (7) at (12.857287543160576em, -11.279145829299912em) {$=$} ; 
\node (8) at (14.571428571428571em, -11.333333333333334em) {$I$} ; 
\node (9) at (17.80952380952381em, -11.333333333333334em) {$I'$} ; 
\node (10) at (24.285714285714285em, -11.333333333333334em) {$B'$} ; 
\node (11) at (21.047619047619047em, -14.571428571428571em) {$A'$} ; 
\node (12) at (21.047619047619047em, -8.095238095238095em) {$A'$} ; 
\node (13) at (17.80952380952381em, -8.095238095238095em) {$A$} ; 
\node (14) at (17.80952380952381em, -14.571428571428571em) {$A$} ; 
\path 
(0) to[fore, black,->, ] node[coordinate](15){} (1) 
(1) to[fore, black,->, ] node[coordinate](16){} (2) 
(0) to[fore, black,->, ] node[coordinate](17){} (3) 
(3) to[fore, black,->, ] node[coordinate](18){} (2) 
(1) to[fore, black,->, cell=0.2, ] node[coordinate](19){} (3) 
(2) to[fore, black,->, ] node[coordinate](20){} (4) 
(3) to[fore, black,->, ] node[coordinate](21){} (5) 
(5) to[fore, black,->, ] node[coordinate](22){} (4) 
(1) to[fore, black,->, ] node[coordinate](23){} (6) 
(6) to[fore, black,->, ] node[coordinate](24){} (4) 
(8) to[fore, black,->, dashed, ] node[coordinate](25){} (9) 
(9) to[fore, black,->, ] node[coordinate](26){} (11) 
(11) to[fore, black,->, ] node[coordinate](27){} (10) 
(9) to[fore, black,->, ] node[coordinate](28){} (12) 
(12) to[fore, black,->, ] node[coordinate](29){} (10) 
(12) to[fore, black,->, cell=0.2, ] node[coordinate](30){} (11) 
(8) to[fore, black,->, ] node[coordinate](31){} (13) 
(13) to[fore, black,->, ] node[coordinate](32){} (12) 
(8) to[fore, black,->, ] node[coordinate](33){} (14) 
(14) to[fore, black,->, ] node[coordinate](34){} (11) 
(6) to[fore, black,->, cell=0.2, ] node[coordinate](35){} (2) 
(2) to[fore, black,->, cell=0.2, ] node[coordinate](36){} (5) 
; 
\path[->] 
(1) edge["${\scriptstyle H_A}$", pos=0.30000000000000004, fore, black,->, ] (6) 
(9) edge["${\scriptstyle i'}$", pos=0.30000000000000004, fore, black,->, ] (12) 
(6) edge["${\scriptstyle F'}$", pos=0.5, fore, black,->, ] (4) 
(12) edge["${\scriptstyle F'}$", pos=0.5, fore, black,->, ] (10) 
(0) edge["${\scriptstyle }$", pos=0.5, fore, black,->, ] (1) 
(1) edge["${\scriptstyle F}$"', pos=0.5, fore, black,->, ] (2) 
(0) edge["${\scriptstyle }$", pos=0.5, fore, black,->, ] (3) 
(3) edge["${\scriptstyle G}$", pos=0.5, fore, black,->, ] (2) 
(1) edge["${\scriptstyle \kappa}$", pos=0.5, fore, black,->, cell=0.2, ] (3) 
(2) edge["${\scriptstyle H_B}$", pos=0.20000000000000004, fore, black,->, ] (4) 
(3) edge["${\scriptstyle H_A}$"', pos=0.5, fore, black,->, ] (5) 
(5) edge["${\scriptstyle G'}$"', pos=0.5, fore, black,->, ] (4) 
(8) edge["${\scriptstyle H_I}$", pos=0.5, fore, black,->, dashed, ] (9) 
(9) edge["${\scriptstyle i'}$"', pos=0.5, fore, black,->, ] (11) 
(11) edge["${\scriptstyle G'}$"', pos=0.5, fore, black,->, ] (10) 
(12) edge["${\scriptstyle \kappa'}$", pos=0.5, fore, black,->, cell=0.2, ] (11) 
(8) edge["${\scriptstyle i}$", pos=0.5, fore, black,->, ] (13) 
(13) edge["${\scriptstyle H_A}$", pos=0.5, fore, black,->, ] (12) 
(8) edge["${\scriptstyle i}$"', pos=0.5, fore, black,->, ] (14) 
(14) edge["${\scriptstyle H_A}$"', pos=0.5, fore, black,->, ] (11) 
(6) edge["${\scriptstyle h_F}$", pos=0.5, fore, black,->, cell=0.2, ] (2) 
(2) edge["${\scriptstyle h_G}$", pos=0.5, fore, black,->, cell=0.2, ] (5) 
; 
\end{tikzpicture}
% END OF GENERATED LATEX
  \]
  \end{lemma}
  \begin{proof}
   By direct application of the universal property (first item).
  \end{proof}
  \begin{lemma}
    Given an inserter $(I\xrightarrow{i}A, κ)$, 
    the 1-cell $i$ is faithful (ptet qu'on n'a pas besoin de ca dans la suite?) and conservative, in the following sense:
    \begin{itemize}
      \item 
        given two 2-cells
        $
        % YADE DIAGRAM diagrams/two-2-cells.json
% GENERATED LATEX
\begin{tikzpicture}[every node/.style={inner sep=2pt,outer sep=0pt,anchor=base,text height=1.2ex, text depth=0.25ex}] 
\node (0) at (10.154834247770763em, -6.450344993954613em) {$X$} ; 
\node (1) at (16.619192305065337em, -6.450344993954613em) {$I$} ; 
\path 
(0) to[fore, black,->, curve={ratio=-0.20000000000000004}, ] node[coordinate](2){} (1) 
(0) to[fore, black,->, curve={ratio=0.30000000000000004}, ] node[coordinate](3){} (1) 
(2) to[fore, black,->, cell=0.2, curve={ratio=-0.30000000000000004}, ] node[coordinate](4){} (3) 
(2) to[fore, black,->, cell=0.2, curve={ratio=0.30000000000000004}, ] node[coordinate](5){} (3) 
; 
\path[->] 
(0) edge["${\scriptstyle x}$", pos=0.5, fore, black,->, curve={ratio=-0.20000000000000004}, ] (1) 
(0) edge["${\scriptstyle x'}$"', pos=0.5, fore, black,->, curve={ratio=0.30000000000000004}, ] (1) 
(2) edge["${\scriptstyle γ'}$", pos=0.5, fore, black,->, cell=0.2, curve={ratio=-0.30000000000000004}, ] (3) 
(2) edge["${\scriptstyle γ}$"', pos=0.5, fore, black,->, cell=0.2, curve={ratio=0.30000000000000004}, ] (3) 
; 
\end{tikzpicture}
% END OF GENERATED LATEX
        $, if $ i· γ = i· γ'$, then $γ = γ'$;
    \item 
      given any 2-cell
      $
      \begin{tikzpicture}[every node/.style={inner sep=2pt,outer sep=0pt,anchor=base,text height=1.2ex, text depth=0.25ex}] 
\node (0) at (5.119047619047619em, -5.119047619047619em) {$X$} ; 
\node (1) at (9.3810970669701em, -5.119047619047619em) {$I$} ; 
\path 
(0) to[fore, black,->, curve={ratio=-0.4}, ] node[coordinate](2){} (1) 
(0) to[fore, black,->, curve={ratio=0.4}, ] node[coordinate](3){} (1) 
(2) to[fore, black,->, cell=0.2, ] node[coordinate](4){} (3) 
; 
\path[->] 
(0) edge["${\scriptstyle x}$", pos=0.5, fore, black,->, curve={ratio=-0.4}, ] (1) 
(0) edge["${\scriptstyle x'}$"', pos=0.5, fore, black,->, curve={ratio=0.4}, ] (1) 
(2) edge["${\scriptstyle γ}$", pos=0.5, fore, black,->, cell=0.2, ] (3) 
; 
\end{tikzpicture}
      $, if $i· γ$ is invertible, then so is $γ$.
    \end{itemize}
  \end{lemma}
  \begin{proof}
    Faihfulness (first item) follows from the second universal property and 
    Remark~\ref{rem:cons-cone}, 
  \end{proof}




%%% Local Variables:
%%% TeX-master: "../main.tex"
%%% End:
