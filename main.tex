%%
%% This is file `sample-sigconf.tex',
%% generated with the docstrip utility.
%%
%% The original source files were:
%%
%% samples.dtx  (with options: `sigconf')
%%
%% IMPORTANT NOTICE:
%%
%% For the copyright see the source file.
%%
%% Any modified versions of this file must be renamed
%% with new filenames distinct from sample-sigconf.tex.
%%
%% For distribution of the original source see the terms
%% for copying and modification in the file samples.dtx.
%%
%% This generated file may be distributed as long as the
%% original source files, as listed above, are part of the
%% same distribution. (The sources need not necessarily be
%% in the same archive or directory.)
%%
%%
%% Commands for TeXCount
%TC:macro \cite [option:text,text]
%TC:macro \citep [option:text,text]
%TC:macro \citet [option:text,text]
%TC:envir table 0 1
%TC:envir table* 0 1
%TC:envir tabular [ignore] word
%TC:envir displaymath 0 word
%TC:envir math 0 word
%TC:envir comment 0 0
%%
%%
%% The first command in your LaTeX source must be the \documentclass
%% command.
%%
%% For submission and review of your manuscript please change the
%% command to \documentclass[manuscript, screen, review]{acmart}.
%%
%% When submitting camera ready or to TAPS, please change the command
%% to \documentclass[sigconf]{acmart} or whichever template is required
%% for your publication.
%%
%%
\documentclass[sigconf,review,anonymous]{acmart}

%%
%% \BibTeX command to typeset BibTeX logo in the docs
\AtBeginDocument{%
  \providecommand\BibTeX{{%
    Bib\TeX}}}

%% Rights management information.  This information is sent to you
%% when you complete the rights form.  These commands have SAMPLE
%% values in them; it is your responsibility as an author to replace
%% the commands and values with those provided to you when you
%% complete the rights form.
\setcopyright{acmlicensed}
\copyrightyear{2018}
\acmYear{2018}
\acmDOI{XXXXXXX.XXXXXXX}

%% These commands are for a PROCEEDINGS abstract or paper.
\acmConference[LICS '24]{Make sure to enter the correct
  conference title from your rights confirmation emai}{July 08--12,
  2024}{Tallinn, Estonia}
%%
%%  Uncomment \acmBooktitle if the title of the proceedings is different
%%  from ``Proceedings of ...''!
%%
%%\acmBooktitle{Woodstock '18: ACM Symposium on Neural Gaze Detection,
%%  June 03--05, 2018, Woodstock, NY}
\acmISBN{978-1-4503-XXXX-X/18/06}


%%
%% Submission ID.
%% Use this when submitting an article to a sponsored event. You'll
%% receive a unique submission ID from the organizers
%% of the event, and this ID should be used as the parameter to this command.
%%\acmSubmissionID{123-A56-BU3}

%%
%% For managing citations, it is recommended to use bibliography
%% files in BibTeX format.
%%
%% You can then either use BibTeX with the ACM-Reference-Format style,
%% or BibLaTeX with the acmnumeric or acmauthoryear sytles, that include
%% support for advanced citation of software artefact from the
%% biblatex-software package, also separately available on CTAN.
%%
%% Look at the sample-*-biblatex.tex files for templates showcasing
%% the biblatex styles.
%%

%%
%% The majority of ACM publications use numbered citations and
%% references.  The command \citestyle{authoryear} switches to the
%% "author year" style.
%%
%% If you are preparing content for an event
%% sponsored by ACM SIGGRAPH, you must use the "author year" style of
%% citations and references.
%% Uncommenting
%% the next command will enable that style.
%%\citestyle{acmauthoryear}


\newcommand{\doublerightarrow}[3]{\ar[#1,shift left=.75ex,"#2"]\ar[#1,shift right=.75ex,swap,"#3"]}
\newcommand{\universalins}[1]{\overline{#1}}
\newcommand{\VInsDiag}{\mathbf{VInsDiag}}
\newcommand{\VInsCone}{\mathbf{VInsCone}}
\newcommand{\Cat}{\mathbf{Cat}}

\usepackage{yade}
\usepackage{tikz-cd}
\usepackage{ebutf8}
\usepackage{import}

% Shortcut for temporary editing
\usepackage{xcolor}
\newcommand{\todo}{\textcolor{red}{\textbf{Todo}}\PackageWarning{todo}{todo}}
\newcommand{\todot}[1]{\textcolor{red}{\textbf{Todo:} #1}\PackageWarning{todo}{todo}}
% lispsum but in red
\usepackage{lipsum}
\newcommand{\todoLipsum}[1]{\todot{\lipsum[#1]}\PackageWarning{todo}{todo}}
% For us to write visible comments to solve
\newcommand{\TL}[1]{\textcolor{blue}{\textbf{TL:} #1}\PackageWarning{todo}{todo}}
\newcommand{\AL}[1]{\textcolor{blue}{\textbf{AL:} #1}\PackageWarning{todo}{todo}}

%%
%% end of the preamble, start of the body of the document source.
\begin{document}

%%
%% The "title" command has an optional parameter,
%% allowing the author to define a "short title" to be used in page headers.
\title{Stuff}

%%
%% The "author" command and its associated commands are used to define
%% the authors and their affiliations.
%% Of note is the shared affiliation of the first two authors, and the
%% "authornote" and "authornotemark" commands
%% used to denote shared contribution to the research.
\author{Ambroise Lafont}
\email{ambroise.lafont@polytechnique.edu}
\orcid{0000-0002-9299-641X}
\affiliation{%
  \institution{Ecole Polytechnique}
  \streetaddress{1 rue Honoré d'Estienne d'Orves}
  \city{Palaiseau}
  \country{France}
  \postcode{91120}
}
\author{Thomas Lamiaux}


%%
%% By default, the full list of authors will be used in the page
%% headers. Often, this list is too long, and will overlap
%% other information printed in the page headers. This command allows
%% the author to define a more concise list
%% of authors' names for this purpose.
% \renewcommand{\shortauthors}{Trovato et al.}

%%
%% The abstract is a short summary of the work to be presented in the
%% article.
\begin{abstract}
  Abstract
\end{abstract}

%%
%% The code below is generated by the tool at http://dl.acm.org/ccs.cfm.
%% Please copy and paste the code instead of the example below.
%%
\begin{CCSXML}
<ccs2012>
 <concept>
  <concept_id>00000000.0000000.0000000</concept_id>
  <concept_desc>Do Not Use This Code, Generate the Correct Terms for Your Paper</concept_desc>
  <concept_significance>500</concept_significance>
 </concept>
 <concept>
  <concept_id>00000000.00000000.00000000</concept_id>
  <concept_desc>Do Not Use This Code, Generate the Correct Terms for Your Paper</concept_desc>
  <concept_significance>300</concept_significance>
 </concept>
 <concept>
  <concept_id>00000000.00000000.00000000</concept_id>
  <concept_desc>Do Not Use This Code, Generate the Correct Terms for Your Paper</concept_desc>
  <concept_significance>100</concept_significance>
 </concept>
 <concept>
  <concept_id>00000000.00000000.00000000</concept_id>
  <concept_desc>Do Not Use This Code, Generate the Correct Terms for Your Paper</concept_desc>
  <concept_significance>100</concept_significance>
 </concept>
</ccs2012>
\end{CCSXML}

\ccsdesc[500]{Do Not Use This Code~Generate the Correct Terms for Your Paper}
\ccsdesc[300]{Do Not Use This Code~Generate the Correct Terms for Your Paper}
\ccsdesc{Do Not Use This Code~Generate the Correct Terms for Your Paper}
\ccsdesc[100]{Do Not Use This Code~Generate the Correct Terms for Your Paper}

%%
%% Keywords. The author(s) should pick words that accurately describe
%% the work being presented. Separate the keywords with commas.
\keywords{Do, Not, Us, This, Code, Put, the, Correct, Terms, for,
  Your, Paper}
%% A "teaser" image appears between the author and affiliation
%% information and the body of the document, and typically spans the
%% page.


%%
%% This command processes the author and affiliation and title
%% information and builds the first part of the formatted document.
\maketitle

\section{Introduction}
Hello
\section{Inserters}
\begin{definition}
  An \emph{inserter diagram $(F,G)$} in a 2-category $ℬ$ is a pair of parallel 1-cells
  $
  \begin{tikzcd}
  A \doublerightarrow{r}{F}{G} & B
  \end{tikzcd}
  $.
  A \emph{cone} of such an inserter diagram is a pair $(X\xrightarrow{x} A,α)$ consisting of a
  1-cell $x$ and a 2-cell $α∶ F · x → G · x$.
  % as in the following diagram:
%   \[
%   OLD YADE DIAGRAM diagrams/inserter.json
% % GENERATED LATEX
% \begin{tikzpicture}[every node/.style={inner sep=5pt,outer sep=0pt,anchor=base,text height=1.2ex, text depth=0.25ex}] 
\node (0) at (7.142857142857143em, -4.285714285714286em) {$X$} ; 
\node (1) at (10em, -1.4285714285714286em) {$A$} ; 
\node (2) at (12.857142857142858em, -4.285714285714286em) {$B$} ; 
\node (3) at (10em, -7.142857142857143em) {$A$} ; 
\path 
(0) to[fore, black,->, ] node[coordinate](4){} (1) 
(1) to[fore, black,->, ] node[coordinate](5){} (2) 
(0) to[fore, black,->, ] node[coordinate](6){} (3) 
(3) to[fore, black,->, ] node[coordinate](7){} (2) 
(1) to[fore, black,->, cell=0.2, ] node[coordinate](8){} (3) 
; 
\path[->] 
(0) edge["${\scriptstyle x}$", pos=0.5, fore, black,->, ] (1) 
(1) edge["${\scriptstyle F}$", pos=0.5, fore, black,->, ] (2) 
(0) edge["${\scriptstyle x}$"', pos=0.5, fore, black,->, ] (3) 
(3) edge["${\scriptstyle G}$"', pos=0.5, fore, black,->, ] (2) 
(1) edge["${\scriptstyle α}$", pos=0.5, fore, black,->, cell=0.2, ] (3) 
; 
\end{tikzpicture}
% % END OF GENERATED LATEX
% \]

A limit of an inserter diagram $(F,G)$, called the \emph{inserter of $F$ and
  $G$}, is a cone $(I\xrightarrow{i}A, κ)$ that is universal, in the following sense:
\begin{itemize}
\item
given any other cone $(X\xrightarrow{x}A, α)$ there exists a unique 1-cell
$X \xrightarrow{\universalins{α}} I$ such that $i∘u = x$ and $\universalins{α} = κ · i$, as summarised below.
\[
% YADE DIAGRAM diagrams/ins-universal-prop.json
% GENERATED LATEX
\begin{tikzpicture}[every node/.style={inner sep=2pt,outer sep=0pt,anchor=base,text height=1.2ex, text depth=0.25ex}] 
\node (0) at (18.333333333333332em, -8.333333333333334em) {$X$} ; 
\node (1) at (21.666666666666668em, -5em) {$A$} ; 
\node (2) at (25em, -8.333333333333334em) {$B$} ; 
\node (3) at (21.666666666666668em, -11.666666666666666em) {$A$} ; 
\node (4) at (28.333333333333332em, -8.333333333333334em) {$=$} ; 
\node (5) at (31.666666666666668em, -8.333333333333334em) {$X$} ; 
\node (6) at (38.333333333333336em, -5em) {$A$} ; 
\node (7) at (41.666666666666664em, -8.333333333333334em) {$B$} ; 
\node (8) at (38.333333333333336em, -11.666666666666666em) {$A$} ; 
\node (9) at (35em, -8.333333333333334em) {$I$} ; 
\path 
(0) to[fore, black,->, ] node[coordinate](10){} (1) 
(1) to[fore, black,->, ] node[coordinate](11){} (2) 
(0) to[fore, black,->, ] node[coordinate](12){} (3) 
(3) to[fore, black,->, ] node[coordinate](13){} (2) 
(1) to[fore, black,->, cell=0.2, ] node[coordinate](14){} (3) 
(6) to[fore, black,->, ] node[coordinate](15){} (7) 
(8) to[fore, black,->, ] node[coordinate](16){} (7) 
(6) to[fore, black,->, cell=0.2, ] node[coordinate](17){} (8) 
(5) to[fore, black,->, dashed, ] node[coordinate](18){} (9) 
(9) to[fore, black,->, ] node[coordinate](19){} (8) 
(5) to[fore, black,->, curve={ratio=-0.20000000000000004}, ] node[coordinate](20){} (6) 
(5) to[fore, black,->, curve={ratio=0.20000000000000004}, ] node[coordinate](21){} (8) 
(9) to[fore, black,->, ] node[coordinate](22){} (6) 
; 
\path[->] 
(0) edge["${\scriptstyle x}$", pos=0.5, fore, black,->, ] (1) 
(1) edge["${\scriptstyle F}$", pos=0.5, fore, black,->, ] (2) 
(0) edge["${\scriptstyle x}$"', pos=0.5, fore, black,->, ] (3) 
(3) edge["${\scriptstyle G}$"', pos=0.5, fore, black,->, ] (2) 
(1) edge["${\scriptstyle α}$", pos=0.5, fore, black,->, cell=0.2, ] (3) 
(6) edge["${\scriptstyle F}$", pos=0.5, fore, black,->, ] (7) 
(8) edge["${\scriptstyle G}$"', pos=0.5, fore, black,->, ] (7) 
(6) edge["${\scriptstyle κ}$", pos=0.5, fore, black,->, cell=0.2, ] (8) 
(5) edge["${\scriptstyle ∃!\universalins{α}}$", pos=0.7, fore, black,->, dashed, ] (9) 
(9) edge["${\scriptstyle i}$"', pos=0.30000000000000004, fore, black,->, ] (8) 
(5) edge["${\scriptstyle x}$", pos=0.5, fore, black,->, curve={ratio=-0.20000000000000004}, ] (6) 
(5) edge["${\scriptstyle x}$"', pos=0.5, fore, black,->, curve={ratio=0.20000000000000004}, ] (8) 
(9) edge["${\scriptstyle i}$", pos=0.5, fore, black,->, ] (6) 
; 
\end{tikzpicture}
% END OF GENERATED LATEX
\]
\item
  given two cones $(X\xrightarrow{x}A,α)$ and $(X\xrightarrow{x'}A,α')$, and a
  natural transformation $γ$ such that Equation~\eqref{eq:mor-cones} holds, there is
  a unique 2-cell $\universalins{α} \xrightarrow{\universalins{γ}}\universalins{α'}$ such that.
  \begin{equation}
    \label{eq:mor-cones}
  % YADE DIAGRAM diagrams/ins-2universal-prop.json
% GENERATED LATEX
\begin{tikzpicture}[every node/.style={inner sep=2pt,outer sep=0pt,anchor=base,text height=1.2ex, text depth=0.25ex}] 
\node (0) at (2.2857142857142856em, -6.857142857142857em) {$X$} ; 
\node (1) at (11.428571428571429em, -6.857142857142857em) {$B$} ; 
\node (2) at (6.857142857142857em, -6.857142857142857em) {$A$} ; 
\node (3) at (6.857142857142857em, -11.428571428571429em) {$A$} ; 
\node (4) at (13.680774400214188em, -8.458880769104516em) {$=$} ; 
\node (5) at (16em, -6.857142857142857em) {$X$} ; 
\node (6) at (23.684848804192em, -6.839611636006773em) {$B$} ; 
\node (7) at (20.571428571428573em, -6.857142857142857em) {$A$} ; 
\node (8) at (20.571428571428573em, -11.428571428571429em) {$A$} ; 
\path 
(0) to[fore, black,->, bend right={-40.10704565915762}, ] node[coordinate](9){} (2) 
(2) to[fore, black,->, ] node[coordinate](10){} (1) 
(0) to[fore, black,->, bend right={34.37746770784939}, ] node[coordinate](11){} (2) 
(0) to[fore, black,->, bend right={17.1887338539247}, ] node[coordinate](12){} (3) 
(3) to[fore, black,->, bend right={11.459155902616464}, ] node[coordinate](13){} (1) 
(2) to[fore, black,->, cell=0.2, ] node[coordinate](14){} (3) 
(5) to[fore, black,->, ] node[coordinate](15){} (7) 
(7) to[fore, black,->, ] node[coordinate](16){} (6) 
(5) to[fore, black,->, bend right={-17.1887338539247}, ] node[coordinate](17){} (8) 
(8) to[fore, black,->, bend right={5.729577951308232}, ] node[coordinate](18){} (6) 
(7) to[fore, black,->, cell=0.2, ] node[coordinate](19){} (8) 
(5) to[fore, black,->, bend right={40.10704565915762}, ] node[coordinate](20){} (8) 
(9) to[fore, black,->, cell=0.2, ] node[coordinate](21){} (11) 
(17) to[fore, black,->, cell=0.2, ] node[coordinate](22){} (20) 
; 
\path[->] 
(0) edge["${\scriptstyle x}$", pos=0.5, fore, black,->, bend right={-40.10704565915762}, ] (2) 
(2) edge["${\scriptstyle F}$", pos=0.5, fore, black,->, ] (1) 
(0) edge["${\scriptstyle x'}$"', pos=0.5, fore, black,->, bend right={34.37746770784939}, ] (2) 
(0) edge["${\scriptstyle x'}$"', pos=0.5, fore, black,->, bend right={17.1887338539247}, ] (3) 
(3) edge["${\scriptstyle G}$"', pos=0.5, fore, black,->, bend right={11.459155902616464}, ] (1) 
(2) edge["${\scriptstyle α'}$", pos=0.5, fore, black,->, cell=0.2, ] (3) 
(5) edge["${\scriptstyle x}$", pos=0.5, fore, black,->, ] (7) 
(7) edge["${\scriptstyle F}$", pos=0.5, fore, black,->, ] (6) 
(5) edge["${\scriptstyle x}$", pos=0.5, fore, black,->, bend right={-17.1887338539247}, ] (8) 
(8) edge["${\scriptstyle G}$"', pos=0.5, fore, black,->, bend right={5.729577951308232}, ] (6) 
(7) edge["${\scriptstyle α}$", pos=0.5, fore, black,->, cell=0.2, ] (8) 
(5) edge["${\scriptstyle x'}$"', pos=0.5, fore, black,->, bend right={40.10704565915762}, ] (8) 
(9) edge["${\scriptstyle γ}$", pos=0.5, fore, black,->, cell=0.2, ] (11) 
(17) edge["${\scriptstyle γ}$", pos=0.5, fore, black,->, cell=0.2, ] (20) 
; 
\end{tikzpicture}
% END OF GENERATED LATEX
  \end{equation}
\end{itemize}
\end{definition}
\section{Application: Translating type systems}
\subsection{Small functorss}
\subsection{Cst}
\subsection{Fibration}
\section{Application: Adjunction}
\subsection{Theorem}
\subsection{Cst}
\section{Application: Equivalence}


\newpage

% Sections
\subimport{sections/}{introduction.tex}
\subimport{sections/}{inserters.tex}
\subimport{sections/}{application_translation.tex}
\subimport{sections/}{application_adjunction.tex}
\subimport{sections/}{application_equivalence.tex}

%%
%% The next two lines define the bibliography style to be used, and
%% the bibliography file.
\bibliographystyle{ACM-Reference-Format}
\bibliography{bib}

% Appendix
\appendix
\subimport{sections/}{appendix_inserters.tex}
\subimport{sections/}{appendix_application_translation.tex}
\subimport{sections/}{appendix_application_adjunction.tex}
\subimport{sections/}{appendix_application_equivalence.tex}



\end{document}
\endinput
%%
%% End of file `sample-sigconf.tex'.
