%%
%% This is file `sample-sigconf.tex',
%% generated with the docstrip utility.
%%
%% The original source files were:
%%
%% samples.dtx  (with options: `sigconf')
%%
%% IMPORTANT NOTICE:
%%
%% For the copyright see the source file.
%%
%% Any modified versions of this file must be renamed
%% with new filenames distinct from sample-sigconf.tex.
%%
%% For distribution of the original source see the terms
%% for copying and modification in the file samples.dtx.
%%
%% This generated file may be distributed as long as the
%% original source files, as listed above, are part of the
%% same distribution. (The sources need not necessarily be
%% in the same archive or directory.)
%%
%%
%% Commands for TeXCount
%TC:macro \cite [option:text,text]
%TC:macro \citep [option:text,text]
%TC:macro \citet [option:text,text]
%TC:envir table 0 1
%TC:envir table* 0 1
%TC:envir tabular [ignore] word
%TC:envir displaymath 0 word
%TC:envir math 0 word
%TC:envir comment 0 0
%%
%%
%% The first command in your LaTeX source must be the \documentclass
%% command.
%%
%% For submission and review of your manuscript please change the
%% command to \documentclass[manuscript, screen, review]{acmart}.
%%
%% When submitting camera ready or to TAPS, please change the command
%% to \documentclass[sigconf]{acmart} or whichever template is required
%% for your publication.
%%
%%
\documentclass[sigconf,review,anonymous]{acmart}

%%
%% \BibTeX command to typeset BibTeX logo in the docs
\AtBeginDocument{%
  \providecommand\BibTeX{{%
    Bib\TeX}}}

%% Rights management information.  This information is sent to you
%% when you complete the rights form.  These commands have SAMPLE
%% values in them; it is your responsibility as an author to replace
%% the commands and values with those provided to you when you
%% complete the rights form.
\setcopyright{acmlicensed}
\copyrightyear{2018}
\acmYear{2018}
\acmDOI{XXXXXXX.XXXXXXX}

%% These commands are for a PROCEEDINGS abstract or paper.
\acmConference[LICS '24]{Make sure to enter the correct
  conference title from your rights confirmation emai}{July 08--12,
  2024}{Tallinn, Estonia}
%%
%%  Uncomment \acmBooktitle if the title of the proceedings is different
%%  from ``Proceedings of ...''!
%%
%%\acmBooktitle{Woodstock '18: ACM Symposium on Neural Gaze Detection,
%%  June 03--05, 2018, Woodstock, NY}
\acmISBN{978-1-4503-XXXX-X/18/06}


%%
%% Submission ID.
%% Use this when submitting an article to a sponsored event. You'll
%% receive a unique submission ID from the organizers
%% of the event, and this ID should be used as the parameter to this command.
%%\acmSubmissionID{123-A56-BU3}

%%
%% For managing citations, it is recommended to use bibliography
%% files in BibTeX format.
%%
%% You can then either use BibTeX with the ACM-Reference-Format style,
%% or BibLaTeX with the acmnumeric or acmauthoryear sytles, that include
%% support for advanced citation of software artefact from the
%% biblatex-software package, also separately available on CTAN.
%%
%% Look at the sample-*-biblatex.tex files for templates showcasing
%% the biblatex styles.
%%

%%
%% The majority of ACM publications use numbered citations and
%% references.  The command \citestyle{authoryear} switches to the
%% "author year" style.
%%
%% If you are preparing content for an event
%% sponsored by ACM SIGGRAPH, you must use the "author year" style of
%% citations and references.
%% Uncommenting
%% the next command will enable that style.
%%\citestyle{acmauthoryear}


\newcommand{\doublerightarrow}[3]{\ar[#1,shift left=.75ex,"#2"]\ar[#1,shift right=.75ex,swap,"#3"]}
\newcommand{\universalins}[1]{\overline{#1}}
\newcommand{\VInsDiag}{\mathbf{VInsDiag}}
\newcommand{\VInsCone}{\mathbf{VInsCone}}
\newcommand{\Cat}{\mathbf{Cat}}

\usepackage{yade}
\usepackage{tikz-cd}
\usepackage{ebutf8}
\usepackage{import}

% Shortcut for temporary editing
\usepackage{xcolor}
\newcommand{\todo}{\textcolor{red}{\textbf{Todo}}\PackageWarning{todo}{todo}}
\newcommand{\todot}[1]{\textcolor{red}{\textbf{Todo:} #1}\PackageWarning{todo}{todo}}
% lispsum but in red
\usepackage{lipsum}
\newcommand{\todoLipsum}[1]{\todot{\lipsum[#1]}\PackageWarning{todo}{todo}}
% For us to write visible comments to solve
\newcommand{\TL}[1]{\textcolor{blue}{\textbf{TL:} #1}\PackageWarning{todo}{todo}}
\newcommand{\AL}[1]{\textcolor{blue}{\textbf{AL:} #1}\PackageWarning{todo}{todo}}

%%
%% end of the preamble, start of the body of the document source.
\begin{document}

%%
%% The "title" command has an optional parameter,
%% allowing the author to define a "short title" to be used in page headers.
\title{Stuff}

%%
%% The "author" command and its associated commands are used to define
%% the authors and their affiliations.
%% Of note is the shared affiliation of the first two authors, and the
%% "authornote" and "authornotemark" commands
%% used to denote shared contribution to the research.
\author{Ambroise Lafont}
\email{ambroise.lafont@polytechnique.edu}
\orcid{0000-0002-9299-641X}
\affiliation{%
  \institution{Ecole Polytechnique}
  \streetaddress{1 rue Honoré d'Estienne d'Orves}
  \city{Palaiseau}
  \country{France}
  \postcode{91120}
}
\author{Thomas Lamiaux}
\email{thomas.lamiaux@ens-paris-saclay.fr}
\orcid{0000-0002-7318-5814}
\affiliation{%
  \institution{Ens Paris-Saclay, Université Paris-Salcay}
  \streetaddress{4 Avenue des Sciences}
  \city{Gif-sur-Ivette}
  \country{France}
  \postcode{91190}
}



%%
%% By default, the full list of authors will be used in the page
%% headers. Often, this list is too long, and will overlap
%% other information printed in the page headers. This command allows
%% the author to define a more concise list
%% of authors' names for this purpose.
% \renewcommand{\shortauthors}{Trovato et al.}

%%
%% The abstract is a short summary of the work to be presented in the
%% article.
\begin{abstract}
  Abstract
\end{abstract}

%%
%% The code below is generated by the tool at http://dl.acm.org/ccs.cfm.
%% Please copy and paste the code instead of the example below.
%%
\begin{CCSXML}
<ccs2012>
 <concept>
  <concept_id>00000000.0000000.0000000</concept_id>
  <concept_desc>Do Not Use This Code, Generate the Correct Terms for Your Paper</concept_desc>
  <concept_significance>500</concept_significance>
 </concept>
 <concept>
  <concept_id>00000000.00000000.00000000</concept_id>
  <concept_desc>Do Not Use This Code, Generate the Correct Terms for Your Paper</concept_desc>
  <concept_significance>300</concept_significance>
 </concept>
 <concept>
  <concept_id>00000000.00000000.00000000</concept_id>
  <concept_desc>Do Not Use This Code, Generate the Correct Terms for Your Paper</concept_desc>
  <concept_significance>100</concept_significance>
 </concept>
 <concept>
  <concept_id>00000000.00000000.00000000</concept_id>
  <concept_desc>Do Not Use This Code, Generate the Correct Terms for Your Paper</concept_desc>
  <concept_significance>100</concept_significance>
 </concept>
</ccs2012>
\end{CCSXML}

\ccsdesc[500]{Do Not Use This Code~Generate the Correct Terms for Your Paper}
\ccsdesc[300]{Do Not Use This Code~Generate the Correct Terms for Your Paper}
\ccsdesc{Do Not Use This Code~Generate the Correct Terms for Your Paper}
\ccsdesc[100]{Do Not Use This Code~Generate the Correct Terms for Your Paper}

%%
%% Keywords. The author(s) should pick words that accurately describe
%% the work being presented. Separate the keywords with commas.
\keywords{Do, Not, Us, This, Code, Put, the, Correct, Terms, for,
  Your, Paper}
%% A "teaser" image appears between the author and affiliation
%% information and the body of the document, and typically spans the
%% page.


%%
%% This command processes the author and affiliation and title
%% information and builds the first part of the formatted document.
\maketitle

\section{Introduction}
Hello
\section{Inserters}
\section{Application: Translating type systems}
\subsection{Small functorss}
\subsection{Cst}
\subsection{Fibration}
\section{Application: Adjunction}
\subsection{Theorem}
\subsection{Cst}
\section{Application: Equivalence}


\newpage

% Sections
\subimport{sections/}{introduction.tex}
\subimport{sections/}{inserters.tex}
\subimport{sections/}{application_translation.tex}
\subimport{sections/}{application_adjunction.tex}
\subimport{sections/}{application_equivalence.tex}

%%
%% The next two lines define the bibliography style to be used, and
%% the bibliography file.
\bibliographystyle{ACM-Reference-Format}
\bibliography{bib}

% Appendix
\appendix
\subimport{sections/}{appendix_inserters.tex}
\subimport{sections/}{appendix_application_translation.tex}
\subimport{sections/}{appendix_application_adjunction.tex}
\subimport{sections/}{appendix_application_equivalence.tex}



\end{document}
\endinput
%%
%% End of file `sample-sigconf.tex'.
