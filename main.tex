\documentclass[acmsmall,anonymous,screen,review]{acmart}

\citestyle{acmauthoryear}


%%
%% \BibTeX command to typeset BibTeX logo in the docs
\AtBeginDocument{%
  \providecommand\BibTeX{{%
    Bib\TeX}}}

%% Rights management information.  This information is sent to you
%% when you complete the rights form.  These commands have SAMPLE
%% values in them; it is your responsibility as an author to replace
%% the commands and values with those provided to you when you
%% complete the rights form.
\setcopyright{acmcopyright}
\copyrightyear{2025}
\acmYear{2025}
\acmDOI{XXXXXXX.XXXXXXX}

%% These commands are for a PROCEEDINGS abstract or paper.
\acmConference[Submitted to POPL '25]{Submitted to POPL
  '25}{2025}{Denver}

%%
%%  Uncomment \acmBooktitle if the title of the proceedings is different
%%  from ``Proceedings of ...''!
%%
%%\acmBooktitle{Woodstock '18: ACM Symposium on Neural Gaze Detection,
%%  June 03--05, 2018, Woodstock, NY}
%\acmPrice{gratos}
\acmISBN{978-1-4503-XXXX-X/18/06}
\newcommand{\doublerightarrow}[3]{\ar[#1,shift left=.75ex,"#2"]\ar[#1,shift right=.75ex,swap,"#3"]}
\newcommand{\universalins}[1]{\overline{#1}}
\newcommand{\VInsDiag}{\mathbf{VInsDiag}}
\newcommand{\VInsCone}{\mathbf{VInsCone}}
\newcommand{\Cat}{\mathbf{Cat}}

\usepackage{yade}
\usepackage{tikz-cd}
\usepackage{ebutf8}
\usepackage{import}
\usepackage{cleveref}

% Shortcut for temporary editing
\usepackage{xcolor}
\newcommand{\todo}{\textcolor{red}{\textbf{Todo}}\PackageWarning{todo}{todo}}
\newcommand{\todot}[1]{\textcolor{red}{\textbf{Todo:} #1}\PackageWarning{todo}{todo}}
% lispsum but in red
\usepackage{lipsum}
\newcommand{\todoLipsum}[1]{\todot{}\lipsum[#1]\PackageWarning{todo}{todo}}
% For us to write visible comments to solve
\newcommand{\TL}[1]{\textcolor{blue}{\textbf{TL:} #1}\PackageWarning{todo}{todo}}
\newcommand{\AL}[1]{\textcolor{blue}{\textbf{AL:} #1}\PackageWarning{todo}{todo}}


% Categories
\DeclareMathOperator{\Set}{Set}
% Categories IS
\DeclareMathOperator{\Mon}{Mon}
\DeclareMathOperator{\Mod}{Mod}
\DeclareMathOperator{\Sig}{Sig}
\DeclareMathOperator{\Model}{Model}
% Categorical constructors
\DeclareMathOperator{\Id}{Id}
\DeclareMathOperator{\Lan}{Lan}
\newcommand{\Adj}[2]{\overset{#1}{\underset{#2}{\leftrightarrows}}}



\newcommand{\mc}[1]{\mathcal{#1}}

\AtEndPreamble{
\theoremstyle{acmdefinition}
\newtheorem{remark}[theorem]{Remark}
}
%%
%% end of the preamble, start of the body of the document source.
\begin{document}

%%
%% The "title" command has an optional parameter,
%% allowing the author to define a "short title" to be used in page headers.
\title{Metatheory of Initial Semantics and Applications}

%%
%% The "author" command and its associated commands are used to define
%% the authors and their affiliations.
%% Of note is the shared affiliation of the first two authors, and the
%% "authornote" and "authornotemark" commands
%% used to denote shared contribution to the research.
\author{Ambroise Lafont}
\email{ambroise.lafont@polytechnique.edu}
\orcid{0000-0002-9299-641X}
\affiliation{%
  \institution{Ecole Polytechnique}
  \streetaddress{1 rue Honoré d'Estienne d'Orves}
  \city{Palaiseau}
  \country{France}
  \postcode{91120}
}
\author{Thomas Lamiaux}
\email{thomas.lamiaux@ens-paris-saclay.fr}
\orcid{0000-0002-7318-5814}
\affiliation{%
  \institution{Ens Paris-Saclay, Université Paris-Salcay}
  \streetaddress{4 Avenue des Sciences}
  \city{Gif-sur-Ivette}
  \country{France}
  \postcode{91190}
}



%%
%% By default, the full list of authors will be used in the page
%% headers. Often, this list is too long, and will overlap
%% other information printed in the page headers. This command allows
%% the author to define a more concise list
%% of authors' names for this purpose.
% \renewcommand{\shortauthors}{Trovato et al.}

%%
%% The abstract is a short summary of the work to be presented in the
%% article.
\begin{abstract}
  Abstract
\end{abstract}

%%
%% The code below is generated by the tool at http://dl.acm.org/ccs.cfm.
%% Please copy and paste the code instead of the example below.
% %%
\begin{CCSXML}
  <ccs2012>
  <concept>
  <concept_id>10003752.10003790.10003798</concept_id>
  <concept_desc>Theory of computation~Equational logic and rewriting</concept_desc>
  <concept_significance>500</concept_significance>
  </concept>
  <concept>
  <concept_id>10003752.10010124.10010131.10010137</concept_id>
  <concept_desc>Theory of computation~Categorical semantics</concept_desc>
  <concept_significance>500</concept_significance>
  </concept>
  </ccs2012>
\end{CCSXML}

\ccsdesc[500]{Theory of computation~Equational logic and rewriting}
\ccsdesc[500]{Theory of computation~Categorical semantics}

\keywords{dependent type theory, category theory}

%% A "teaser" image appears between the author and affiliation
%% information and the body of the document, and typically spans the
%% page.


%%
%% This command processes the author and affiliation and title
%% information and builds the first part of the formatted document.
\maketitle

% \section{Introduction}
% Hello \cite{Adamek}
% \section{Inserters}
% \section{Application: Translating type systems}
% \subsection{Small functorss}
% \subsection{Cst}
% \subsection{Fibration}
% \section{Application: Adjunction}
% \subsection{Theorem}
% \subsection{Cst}
% \section{Application: Equivalence}

% Sections
\subimport{sections/}{introduction.tex}
\subimport{sections/}{inserters.tex}
\subimport{sections/}{vertical-inserters.tex}
\subimport{sections/}{inserters-retour.tex}
\subimport{sections/}{theorem.tex}
\subimport{sections/}{application_translation.tex}
\subimport{sections/}{application_adjunction.tex}
\subimport{sections/}{application_equivalence.tex}

%%
%% The next two lines define the bibliography style to be used, and
%% the bibliography file.
\bibliographystyle{ACM-Reference-Format}
\bibliography{bib}

% Appendix
\appendix
\subimport{sections/}{appendix_inserters.tex}
\subimport{sections/}{appendix_application_translation.tex}
\subimport{sections/}{appendix_application_adjunction.tex}
\subimport{sections/}{appendix_application_equivalence.tex}



\end{document}
\endinput
%%
%% End of file `sample-sigconf.tex'.
